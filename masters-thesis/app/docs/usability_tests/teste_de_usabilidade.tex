% SPDX-License-Identifier: CC-BY-4.0
%
% Copyright (c) 2023 Nelson Vieira
%
% @author Nelson Vieira <2080511@student.uma.pt>
% @license CC-BY-4.0 <https://creativecommons.org/licenses/by/4.0/legalcode.txt>
%
% Commands and layout from https://github.com/MIMBCD-UI/sus
% by Francisco Maria Calisto, MIT License
%
\documentclass[12pt,a4paper]{article}
\usepackage{cite}
\usepackage{hyperref}
\usepackage{graphicx}
\usepackage[T1]{fontenc}
\usepackage[utf8]{inputenc}
\usepackage{enumitem}
\usepackage{array,tabularx}

\setlist[enumerate,1]{labelindent=0pt, leftmargin=0pt}
\pagestyle{empty}

\newcolumntype{P}{>{\centering\arraybackslash}p{0.50cm}}
\newcolumntype{L}{>{\raggedright\arraybackslash}m{0.25\textwidth}}
\newcolumntype{R}{>{\raggedleft\arraybackslash}m{0.25\textwidth}}

\newcommand{\usetbl}{%
    \begin{tabular}{@{}|*7{P|}@{}}
        \hline
        1 & 2 & 3 & 4 & 5 & 6 & 7 \\
        \hline
    \end{tabular}
}

\newcommand\prop[1]{%
    \item
    \parbox[t]{0.5\textwidth}{#1}%
    \qquad
    \parbox[t]{0.5\textwidth}{\usetbl}%
}

\graphicspath{{assets/images/}}

\hypersetup{
    pdftitle={Usability Tests},
    pdfauthor={Nelson Vieira},
    pdfsubject={Usability Tests},
    pdfkeywords={privacy application, internet of things, user empowerment, privacy assistant, usability test},
    colorlinks=true,
    linkcolor=blue,
    filecolor=magenta,
    urlcolor=cyan,
    citecolor=black,
}
\def\myversion{1.0}
\date{}

\begin{document}

\noindent
\hfill \includegraphics[width=4.5cm]{assets/images/uma_logo.png} \\
\normalsize
\textbf{Teste de Usabilidade do Sistema} \hfill \textbf{Universidade da Madeira} \\
\footnotesize
\hyperlink{project_link}{Aplicação para dissertação de mestrado} \hfill Faculdade de Ciências Exatas e da Engenharia \\

\section*{Informação}

Este teste de usabilidade é realizado no âmbito da tese de Mestrado em Engenharia
Informática da Universidade da Madeira, da autoria de Nelson Vieira, tem como
objectivo recolher informação sobre a usabilidade da aplicação, design, bugs ou
outras qualidades indesejáveis, de forma a torná-la mais robusta e fácil de utilizar.

\vspace{1cm}

\begin{tabular}{|m{10cm}|m{1cm}|}
    \hline
    Aceito participar neste teste & \\
    \hline
\end{tabular}

\vspace{1cm}

\textbf{Versão da App}

\clearpage

\section*{Tarefas}

\begin{enumerate}
    \item Ir para a página de dispositivos IoT \\
    \hspace*{0.59\textwidth}%
    \begin{tabularx}{0.1\textwidth}{@{}LR@{}}
        \textbf{\small Muito difícil} & \textbf{\small Muito fácil}
    \end{tabularx}
    \begin{enumerate}
        \prop{No geral, esta tarefa foi?}
    \end{enumerate}
    \item Procurar mais informações sobre um dispositivo
    \begin{enumerate}
        \prop{No geral, esta tarefa foi?}
    \end{enumerate}
    \item Na página inicial, procurar mais informações sobre um dispositivo
    \begin{enumerate}
        \prop{No geral, esta tarefa foi?}
    \end{enumerate}
    \item Procurar mais informações sobre esta aplicação
    \begin{enumerate}
        \prop{No geral, esta tarefa foi?}
    \end{enumerate}
    \item Criar uma conta
    \begin{enumerate}
        \prop{No geral, esta tarefa foi?}
    \end{enumerate}
    \item Procurar mais informações sobre privacidade e Internet das Coisas na aplicação
    \begin{enumerate}
        \prop{No geral, esta tarefa foi?}
    \end{enumerate}
    \item Procurar um tutorial sobre como adicionar um dispositivo
    \begin{enumerate}
        \prop{No geral, esta tarefa foi?}
    \end{enumerate}
    \item Adicionar um dispositivo à aplicação
    \begin{enumerate}
        \prop{No geral, esta tarefa foi?}
    \end{enumerate}
    \item Actualizar um dispositivo na aplicação
    \begin{enumerate}
        \prop{No geral, esta tarefa foi?}
    \end{enumerate}
\end{enumerate}

\clearpage

\section*{Aferição da Usabilidade}

\hspace*{0.525\textwidth}%
\begin{tabularx}{0.5\textwidth}{@{}LR@{}}
    \textbf{Discordo} & \textbf{Concordo} \\
    \textbf{totalmente} & \textbf{totalmente} \\
\end{tabularx}

\begin{enumerate}
    \prop{Penso que gostaria de utilizar este sistema com frequência}

    \prop{Achei o sistema desnecessariamente complexo}

    \prop{Achei que o sistema era fácil de utilizar}

    \prop{Penso que necessitaria do apoio de uma pessoa técnica para poder utilizar este sistema}

    \prop{Considero que as várias funções deste sistema estão bem integradas}

    \prop{Achei que havia demasiada incoerência neste sistema}

    \prop{Imagino que a maioria das pessoas aprenderia a utilizar este sistema muito rapidamente}

    \prop{Achei o sistema muito complicado de utilizar}

    \prop{Senti-me muito confiante ao utilizar o sistema}

    \prop{Precisava de aprender muitas coisas antes de poder avançar com este sistema}
\end{enumerate}

% \clearpage

% \section*{Final Remarks}

% \vspace{1cm}

% \par After using the app, do you think it sparked your curiosity to learn more about privacy in Internet of Things systems? \\

% \vspace{3cm}

% \par After using the app, do you think it sparked your curiosity to learn more about digital privacy in general? \\

% \vspace{3cm}

% \par Can you easily identify which device collects what data just by seeing them in the map or on the devices page? \\

% \vspace{3cm}

% \par What would you change on the app to make it easier to use? \\

% \vspace{3cm}

% \par Do you have any other suggestion? \\

% \bibliographystyle{IEEEtran}
% \bibliography{assets/references}

\end{document}
