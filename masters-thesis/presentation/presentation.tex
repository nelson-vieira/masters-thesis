% SPDX-License-Identifier: CC-BY-4.0
%
% Copyright (c) 2023 Nelson Vieira
%
% @author Nelson Vieira <nelson0.vieira@gmail.com>
% @license CC-BY-4.0 <https://creativecommons.org/licenses/by/4.0/legalcode.txt>
\documentclass[xcolor={svgnames},compress,aspectratio=169]{beamer}
\usetheme{Berlin}
\usecolortheme{dolphin}

\setbeamercolor*{structure}{bg=Azure,fg=MidnightBlue!50!black}

\setbeamercolor*{palette primary}{use=structure,fg=structure.bg,bg=structure.fg}
\setbeamercolor*{palette secondary}{use=structure,fg=structure.fg,bg=structure.bg}
\setbeamercolor*{palette tertiary}{use=structure,fg=structure.fg,bg=GhostWhite}
\setbeamercolor*{palette quaternary}{fg=white,bg=black}

\setbeamercolor{section in head/foot}{parent=palette primary} % Outer section of header/footer
\setbeamercolor{subsection in head/foot}{parent=palette secondary} % Inner section of header/footer

\setbeamercolor{titlelike}{parent=palette tertiary} % Main titles
\setbeamercolor{frametitle}{parent=palette tertiary,bg=GhostWhite!50}

\setbeamercolor{section in toc}{fg=darkgray,bg=Azure} % Table of contents sections
\setbeamercolor{subsection in toc}{fg=darkgray,bg=Azure} % Table of contents subsections
% \setbeamercolor{alerted text}{use=structure,fg=structure.fg!50!black!80!black}

% \setbeamertemplate{navigation symbols}{} % Hides navigation buttons at the bottom
% \setbeamertemplate{headline}{} % Hides navigation bar at the top

\setbeamercovered{transparent}

\setbeamertemplate{caption}[numbered]

% \usepackage{pgfpages}
% \pgfpagesuselayout{4 on 1}[a4paper,border shrink=5mm]

\usepackage[utf8]{inputenc}
\usepackage{adjustbox}
\usepackage{xcolor,colortbl}
\usepackage[all]{xy}
\usepackage{tikz}
\usetikzlibrary{mindmap,backgrounds}
\usepackage{graphicx}
\usepackage{multicol}
% Advanced table functions
\usepackage{tabularx,ragged2e}
\usepackage{booktabs}
% Listings extension
\usepackage{listings}
\usepackage{transparent}
\usepackage{svg}
\setsvg{inkscape = "C:/Program Files/Inkscape/bin/inkscape.exe"}
\svgsetup{inkscapepath=svgsubdir}
\def\myversion{0.1}

\title[Privacy in the Internet of Things: Fostering User Empowerment through Digital Literacy]{Master's Dissertation \\ {\normalsize Privacy in the Internet of Things: Fostering User Empowerment through Digital Literacy}}
% \subtitle{Empowering Users' Privacy Rights in the Internet of Things}
\author{\href{mailto:2080511@student.uma.pt}{Nelson Vieira}
\\ \and \textcolor{gray}{Orientation} \href{mailto:mary.barreto@staff.uma.pt}{Mary Barreto}
}
\institute[\href{https://www.uma.pt/}{University of Madeira}]{University of Madeira\\Faculty of Exact Sciences and Engineering}
\date{{\scriptsize Last Update: \today}}

\setbeameroption{hide notes}

\makeatletter
    \newenvironment{withoutheadline}{
        \setbeamertemplate{headline}[default]
        \def\beamer@entrycode{\vspace*{-\headheight}}
    }{}
\makeatother

\begin{document}

\begin{withoutheadline}
    \begin{frame}
        \centering\includegraphics[width=90pt]{../thesis/assets/images/uma_logo.png}
        \maketitle
    \end{frame}
\end{withoutheadline}

\begin{frame}{Table of Contents}
    % Use hideallsubsections for longer presentations
    % \tableofcontents[hideallsubsections]
    \begin{multicols}{2}
        \tableofcontents
    \end{multicols}
\end{frame}

\section{Introduction}

\begin{frame}{Introduction}
    Internet of Things (IoT) devices are everywhere. These devices
    create new ways of collecting and process personal data from users and
    non-users. Most end users are not even aware or have little control over
    the information that is being collected by these systems.

    This work takes an holistic approach to this problem by doing:
    \begin{itemize}
        \item<1-> Systematic literature review;
        \item<2-> A survey;
        \item<3-> A mobile application.
    \end{itemize}
\end{frame}

\note{
    Some notes here.
}

\begin{frame}
    \begin{multicols}{2}
        \centering
        {\footnotesize What is privacy?}
        \begin{figure}
            \centering\includegraphics[width=155pt]{assets/images/privacy_opinions.png}\\
            \textcolor{gray}{{\tiny \textcopyright \href{https://xkcd.com/1269/}{Randall Munroe}, \href{https://creativecommons.org/licenses/by-nc/2.5/}{CC BY-NC 2.5 License}}}
        \end{figure}

        \columnbreak
        \centering
        \vspace*{\fill}
        Privacy $\ne$ Security
        \vspace*{\fill}
    \end{multicols}
\end{frame}

\note{
    Some notes here.
}

\section{State of the Art}

\begin{frame}{State of the Art}
\end{frame}

\note{
    Some notes here.
}

\subsection{Literature Approaches}

\begin{frame}{Literature Approaches}
\end{frame}

\note{
    Some notes here.
}

\section{Methodology}

\subsection{Survey}

\begin{frame}[shrink]{Survey}
    \begin{multicols}{2}
        86 Questions
        \begin{itemize}
            \item General knowledge and attitudes towards privacy
            \item Disposition for sharing personal information
            \item Privacy concerns
            \item Current online habits and practices
            \item Profile identification
            \item Knowledge and habits regarding the Internet of Things
            \item Demographic data
        \end{itemize}

        \columnbreak
        \vspace*{\fill}
        \begin{figure}
            \centering
            \includegraphics[width=45pt]{assets/images/forms.png}
            % \caption{Google Forms}
        \end{figure}
        \vspace*{\fill}
    \end{multicols}
\end{frame}

\note{
    Some notes here.
}

\subsection{Application}

\begin{frame}{Application}
    \begin{multicols}{2}
        What can the application do?
        \begin{itemize}
            \item Show the geolocation of the IoT devices;
            \item Information about the devices, like category, collection purpose, stored time, owner, etc.;
            \item Information about IoT privacy;
            \item Addition and editing of device's information.
        \end{itemize}

        \columnbreak
        \begin{figure}
            \centering\includegraphics[width=120pt]{assets/images/flutter.png}
            % \caption{Flutter framework}
        \end{figure}
        \begin{figure}
            \centering
            \includesvg[width=140pt]{assets/images/firebase}
            % \caption{Firebase cloud backend}
        \end{figure}
    \end{multicols}
\end{frame}

\note{
    Some notes here.
}

{
\usebackgroundtemplate{{\transparent{0.1}\hspace*{0.2cm}\includegraphics[keepaspectratio,width=7.5cm]{../app/assets/images/icon.png}}}
\begin{frame}
    \centering
    \vspace*{\fill}
    {\Large Demonstration}
    \vspace*{\fill}
\end{frame}
}

\section{Conclusion and Future Work}

\subsection{Future Work}

\begin{frame}{Future Work}
    \begin{itemize}
        \item[$\bullet$]
        Privacy literacy in IoT systems;
        \item[$\bullet$]
        Application of privacy in the design/development of IoT systems;
        \item[$\bullet$]
        Interoperability standards;
        \item[$\bullet$]
        User-centric approaches to IoT privacy.
    \end{itemize}
\end{frame}

\note{
    Some notes here.
}

\subsection{Conclusion}

\begin{frame}{Conclusion}
    This work contributed to the overall body of research by compiling and reviewing
    other works with the perspective of privacy as a distinct subject matter rather than
    an extension of security, as many publications imply. The survey conducted
    on the perception of individuals on privacy in IoT systems portrays the majority
    viewpoint of portuguese people, since 60\% of participants were portuguese. Additionally,
    a mobile application was developed and tested revealing that it performs as it was
    initially designed and envisioned since it reaches its purpose on its own without having
    to rely on additional platforms.
\end{frame}

\note{
    Some notes here.
}

\begin{frame}{Questions and Comments}
    Thank you for your attention. Any questions?
\end{frame}

\note{
    Some notes here.
}

\section*{References}

\begin{frame}[allowframebreaks]
    \bibliographystyle{IEEEtran}
    \bibliography{assets/references}
\end{frame}

\end{document}
