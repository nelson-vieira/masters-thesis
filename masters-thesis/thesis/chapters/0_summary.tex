% SPDX-License-Identifier: CC-BY-4.0
%
% Copyright (c) 2023 Nelson Vieira
%
% @author Nelson Vieira <nelson0.vieira@gmail.com>
% @license CC-BY-4.0 <https://creativecommons.org/licenses/by/4.0/legalcode.txt>
\chapter*{Resumo}
\justify

Os dispositivos da Internet das Coisas estão por todo o lado, desde
o nascimento da computação ubíqua que se prevê que a vida quotidiana
do ser humano contenha milhões de dispositivos que controlam todos os
aspectos da nossa vida. Hoje em dia, temos carros inteligentes, casas
inteligentes, cidades inteligentes, dispositivos portáteis, entre
outros, que utilizam vários tipos de dispositivos e vários tipos de
redes para comunicar. Estes dispositivos criam novas formas de recolha
e tratamento de dados pessoais de utilizadores e não utilizadores.
A maioria dos utilizadores finais nem sequer tem conhecimento ou tem
pouco controlo sobre a informação que está a ser recolhida por
estes sistemas. Este trabalho adopta uma abordagem holística a este
problema, começando por realizar uma revisão da literatura para
compilar as soluções actuais, os desafios e as oportunidades de
investigação futura, realizando depois um inquérito para saber mais
sobre o conhecimento geral dos indivíduos sobre privacidade, IoT e
hábitos \textit{online} e, finalmente, com base na informação recolhida,
é proposta uma aplicação que fornece aos utilizadores informações
sobre os dispositivos que estão próximos e como proteger os dados
que não querem partilhar com esses dispositivos.
Esta aplicação é capaz de detetar que tipo de dispositivos estão nas
proximidades, que tipo de dados são recolhidos por esses dispositivos
e apresentar opções de privacidade ao utilizador sempre que possível,
com o objetivo de fornecer aos indivíduos uma ferramenta para tomarem
decisões informadas sobre os seus dados privados.

\keywords{privacidade \and Internet das Coisas \and computação ubíqua \and desafios na IoT \and literacia digital}
