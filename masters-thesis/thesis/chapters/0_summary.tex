% SPDX-License-Identifier: CC-BY-SA-4.0
%
% Copyright (c) 2023 Nelson Vieira
%
% @author Nelson Vieira <nelson0.vieira@gmail.com>
% @license CC-BY-SA-4.0 <https://creativecommons.org/licenses/by-sa/4.0/legalcode.txt>

\chapter*{Summary}
\justify

Internet of things devices are everywhere, since the birth of ubiquitous
computing that human every day life is envisioned containing millions of
devices that control every aspect of our lives. Today we have smart cars,
smart houses, smart cities, wearables among other things that use various
types of devices and various types of networks to communicate. These devices
create new ways of collecting and process personal data from users and
non-users. Most end users are not even aware or have little control over
the information that is being collected by these systems. This work takes
an holistic approach to this problem by first doing a literature review,
then conducting a survey to learn more about the general knowledge of the
public, and finally, based on the information gathered, a system is proposed
that gives users information about the devices that are nearby and how
to protect the data that they do not want to share with these devices, this
system is capable of detecting what type of devices are nearby, what kind
of data is collected by these devices, show privacy choices to the user
when it is possible to do so and what can be done to protect unwanted data
from being collected.

\keywords{privacy \and Internet of Things \and ubiquitous computing \and privacy assistant}
