% SPDX-License-Identifier: CC-BY-4.0
%
% Copyright (c) 2023 Nelson Vieira
%
% @author Nelson Vieira <nelson0.vieira@gmail.com>
% @license CC-BY-4.0 <https://creativecommons.org/licenses/by/4.0/legalcode.txt>
\chapter*{Resumo}
\justify

Os dispositivos da Internet das coisas estão por todo o lado, desde
o nascimento da computação ubíqua que se prevê que a vida quotidiana
do ser humano contenha milhões de dispositivos que controlam todos os
aspectos da nossa vida. Hoje em dia, temos carros inteligentes, casas
inteligentes, cidades inteligentes, dispositivos portáteis, entre
outros, que utilizam vários tipos de dispositivos e vários tipos de
redes para comunicar. Estes dispositivos criam novas formas de recolha
e tratamento de dados pessoais de utilizadores e não utilizadores.
A maioria dos utilizadores finais nem sequer tem conhecimento ou tem
pouco controlo sobre as informações que estão a ser recolhidas por
estes sistemas. Este trabalho adopta uma abordagem holística a este
problema, começando por fazer uma revisão da literatura, depois conduzindo
um inquérito para saber mais sobre o conhecimento geral do público e,
finalmente, com base na informação recolhida, é proposto um sistema
que dá aos utilizadores informações sobre os dispositivos que estão
nas proximidades e como proteger os dados que não querem partilhar
com esses dispositivos. Este sistema é capaz de detetar que tipo de
dispositivos estão nas proximidades, que tipo de dados são recolhidos
por esses dispositivos, mostrar opções de privacidade ao utilizador
quando é possível fazê-lo e o que pode ser feito para proteger dados
indesejados de serem recolhidos.

\keywords{privacidade \and Internet das Coisas \and computação ubíqua \and desafios na IoT \and literacia digital}
