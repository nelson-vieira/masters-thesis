% SPDX-License-Identifier: CC-BY-4.0
%
% Copyright (c) 2023 Nelson Vieira
%
% @author Nelson Vieira <nelson0.vieira@gmail.com>
% @license CC-BY-4.0 <https://creativecommons.org/licenses/by/4.0/legalcode.txt>
\chapter*{Abstract}
\justify

Internet of Things devices are everywhere, since the birth of ubiquitous
computing, human everyday life is expected to contain millions of
devices that control every aspect of our lives. Today we have smart vehicles,
smart houses, smart cities, wearables among other things that use various
types of devices, and various types of networks to communicate. These devices
create new ways of collecting and processing personal data from users, and
non-users. Most end users are not even aware or have little control over
the information that is being collected by these systems. This work takes
a holistic approach to this problem by first conducting a literature review
to compile current solutions, challenges and future research opportunities.
Then conducting a survey to learn more about the general knowledge of
individuals about privacy, IoT and online habits, and finally, based on the
information gathered, a mobile application is proposed that gives users
information about nearby devices, and how
to protect the data that they do not want to share with them.
User testing revealed that participants valued having access to more
information about privacy related terms. This
application is capable of detecting what type of devices are nearby, what kind
of data is collected by these devices, and displaying privacy options to the user,
when it is possible to do so, with the goal of providing individuals a tool to make
informed decisions about their private data.

\keywords{privacy \and Internet of Things \and ubiquitous computing \and challenges in IoT \and digital literacy}
