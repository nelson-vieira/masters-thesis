% SPDX-License-Identifier: CC-BY-4.0
%
% Copyright (c) 2023 Nelson Vieira
%
% @author Nelson Vieira <nelson0.vieira@gmail.com>
% @license CC-BY-4.0 <https://creativecommons.org/licenses/by/4.0/legalcode.txt>
\section{Discussion}

To answer the research question \textbf{RQ3}, the questionnaire makes it clear that there is
a general lack of digital literacy, especially when it comes to IoT.
This still being a new terminology/technology and only quickly expanding
on the last decade, the people that have the most knowledge are the
ones working in areas related to it. This survey also helps to demystify
the privacy paradox.

The application by itself does not provide any formal privacy protection
on IoT devices, but users can use it to better their understanding of IoT
and in some cases make privacy choices regarding a Iot device.

As concluded in the survey and users do not have a great deal of literacy
regarding IoT, they do have some privacy literacy though, so
to answer the research question \textbf{RQ4}, this application servers to increase users
privacy literacy by giving them tools to know what kind of devices
are around them, what these devices do and give them more information
regarding IoT in general and IoT privacy, so that users may make
well-informed choices.

Some level of theoretical saturation \cite{low2019pragmatic} was reached with the use of the questionnaire
and the usability tests, i.e. it was extracted the most amount of information
possible from the participants on this topic of privacy on IoT systems, since
when doing the usability tests most participants also completed the questionnaire.

This application does not claim to be the best in addressing IoT privacy issues but it improves
on the previous works by not only focusing on privacy choices but also
providing privacy literacy to the end user, which is the ultimate goal
of the application. While existing works are fragmented on their approach,
this application offers a more centralized way by allowing users to complete
all tasks using it. Users can also contribute to the application itself
through various methods, the most straightforward way is using it and
adding new IoT devices to the application's database, they can also leave
feedback for improving the application on distribution platforms. Because
this application is distributed as free and open source software, it
is possible for users to the development of the
application itself by contributing with code or raising questions about
features or bugs, it also
can demystify privacy concerns about the application itself because
anyone can inspect the source code.
