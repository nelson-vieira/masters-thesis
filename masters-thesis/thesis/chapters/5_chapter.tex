% SPDX-License-Identifier: CC-BY-4.0
%
% Copyright (c) 2023 Nelson Vieira
%
% @author Nelson Vieira <nelson0.vieira@gmail.com>
% @license CC-BY-4.0 <https://creativecommons.org/licenses/by/4.0/legalcode.txt>

\section{Methodology}

The overall work will be comprised of two phases which will be described
in the following paragraphs. Phase one mainly described throughout this
paper, focuses on collecting the state of the art in terms of the most relevant
topics, from which main privacy concepts were selected to be explored in
the stage 1 of Phase 2 with the preparation of a questionnaire to collect
user perceptions regarding privacy and topics collected in the systematic
literature review. The second stage of Phase 2 consists in developing an
application, partially based on the information generated by the survey,
that can identify what sort of devices are around, what kind of data is
gathered by these devices, present privacy options to the user when available,
and what can be done to prevent undesirable data from being collected.
\par
The Phase 1 Systematic Literature Review gathered the most relevant papers
discussing methodologies and techniques for the protection of users' privacy
data with special focus on IoT systems. For this SLR, this paper considered
focusing only on papers from the last 12 years, from 2010 until 2022, since
papers before then become out of date with the evolution of technology.
In this SLR, it was reviewed 54 papers published in top computer science,
security, privacy and software engineering outlets.

This paper followed Keshav's three-pass approach \cite{KeshavHow} when choosing
which papers to read fully and which ones to ignore, first the title would
be read, then the abstract, the introduction and conclusion and briefly
skim the rest of the paper and then decide if it was worth reading any further,
the focal point in this phase was answering the following question: does
the paper present a new methodology or interesting angle to tackle users'
privacy concerns? Only then the document would be read in its entirety while
ignoring any tables, figures, images or graphs. If the paper failed to present
any interesting idea, approach, or technique it would be discarded, but
if not, it would be read carefully from the beginning again in order to
fully understand what it presents. Having collected the major findings,
this work then aims to conduct a throughout study split in several stages
and around the specific research questions which will be explored in each
phase. For that matter, the research questions listed are:

\vspace{5mm}
\textbf{Phase 1:} \\
% \vfill

\textbf{RQ1:} What approaches are being considered for privacy issues in
IoT in the currently available literature?

\textbf{RQ2:} What are user perceptions on online privacy? \\

% \vspace{5mm}
\textbf{Phase 2:} \\
% \vfill

\textbf{RQ3:}
% What IoT-related tools are available that empower users to
% protect their privacy rights? OR
How to empower users to protect their privacy rights?

\textbf{RQ4:} What issues are prevalent in IoT that make it difficult to
address privacy and security problems?
\vspace{5mm}

The second phase will be evaluated on two stages, the first one consists
on doing a study on people's general privacy concerns, while using and interacting
with IoT devices. This study will abide on preparing a questionnaire to
assess general user's knowledge on privacy concepts, their habits and concerns,
their understanding of privacy rights, and what they do to safeguard those
rights. The goal of this study is to both understand the privacy paradox
and collect data on their proposal to address privacy issues with regard
to IoT devices.

% Having collected the major findings, this work then aims to conduct a throughout
% study split in several stages and around the following research questions:

% \textbf{RQ1:} What approaches are being considered for privacy issues in
% IoT in the currently available literature?

% \textbf{RQ2:} What IoT-related tools are available that empower users to
% protect their privacy rights? OR How to empower users to protect their privacy
% rights?

% \textbf{RQ3:} What issues are prevalent in IoT that make it difficult to
% address privacy and security problems?

% The proposed methodology is composed of two phases, the first phase consists
% on doing a study on people's general privacy concerns while using and interacting
% with IoT devices. This study will consist on preparing a questionnaire to
% assess general user's knowledge on privacy concepts, their habits and concerns,
% their understanding of privacy rights, and what they do to safeguard those
% rights. The goal of this study is to both understand the privacy paradox
% and collect data on their proposal to address privacy issues with regard
% to IoT devices. The second phase consists in developing an application, partially
% based on the information generated by the survey, that can identify what sort
% of devices are around, what kind of data is gathered by these devices, present
% privacy options to the user where available, and what can be done to prevent
% undesirable data from being collected.

% The second phase consists
% in doing an application that can detect IoT devices nearby the user with
% at least a 10 meters radius. The application should do the following when
% detecting a device:
% 1. it should show some information about the device;
% 2. it should categorize the device;
% 3. it should provide the user with privacy options, if the device allows the
% user to decline data harvesting.
% This application at first sight might appear to be a mere privacy assistant
% but it's not, because IoT assistants merely choose what privacy options the
% user first sets and maintains it for every other application that the user
% might use. The proposed app doesn't have the objective to conform to the user's
% preferred privacy choices, it merely informs the user about nearby IoT devices
% and can provide the user with privacy options. But the main objective is creating
% awareness in individuals about the various devices that are around and make
% the user questions their choices.

\subsection{Stage 1: User perceptions}

This study aims to understand people's perception of IoT and their privacy
practices online. It also serves to demystify the privacy paradox and also
to help provide a solution to the privacy issue in IoT. The questionnaire
consists of 92 questions divided into 7 sections to access users' knowledge,
it follows a kind of narrative, the first section being general privacy
questions then about the predisposition to data sharing, to concerns with
privacy then about daily digital routines, then about profile identification,
and then about IoT general knowledge before a section about non-identifiable
demographic data. The scale that is used in the questionnaire is based on
the work of Philip K. Masur \cite{masur2018situational}. Great care is taken
when it comes to this survey's data collection, in order to not identify
any individual or group of individuals, for instance, when it comes to differential
privacy, any data that might identify someone will not be disclosed, even
though the data might suffer from some inaccuracy because of this.

This survey was partially based in a study done in the Philippines by the
government in the context of their privacy act of 2012 \cite{Philippine2022Conduct},
this was the second survey done on the country's population. It was also
inspired by Alves's master's thesis \cite{alves2021}, which was about citizen's
perception about privacy in the wake of GDPR.

This survey was done through the internet, it was created in Google Forms,
this way it is guaranteed to reach the most people possible, besides Google
Forms itself, it will be used other online venues for distribution and even
printing.

\subsection{Stage 2: Study in Context, an Application}

This work proposes an application that gives users information about IoT
devices in their surroundings like the type of information these devices
collect and what privacy options are available. This application will be
developed for mobile phones because it is the most used device that people
take everywhere they go, and because the application will use georeferencing
to show the location of the IoT devices. The main objective of this application
is to give users another option in order to protect their private data.
The application will show the geolocation of the IoT devices, what type
of device it is, what type of data is being collect by the device. The application
will not detect the devices by itself, this will be done by the users themselves,
in the first few iterations of this application it was proposed that the
application itself would detect the devices and would categorize what type
of device it was and what type of data it was collecting but it was discovered
that this approach was too complex and so it was not feasible to do with
the constraints of this paper. The application will be developed with Flutter,
other options could be React Native or a progressive web application, but
Flutter uses ahead of time and just in time compilation with Dart as it
is programming language while React Native uses the Javascript programming
language that was never created for mobile programming so it uses a bridge
to convert Javascript to native components for Android or iOS. Flutter has
better performance and as such it is the chosen framework for this application.

\section{Current Stage of the Work}

The preliminary results of the study, based on 10 responses, show that everyone
agrees that privacy is important to them and some people know that they
should not share their personal information with anyone they do not trust
(like clicking on random urls or using unprotected websites/software), but
most of them think that privacy and security are the same concept, most
respondents also do not read privacy notices but accept them to access the
information they want to get to, most respondents use their devices mostly
to access social networks and for work, when it comes to IoT, there is a
dissonance between knowing the term and using devices like smart watches
or RFID enabled devices, from the respondents that answered yes to using
IoT devices most use because of work. It is also noted that most respondent
have a background in engineering, so the responses are skewed. As a result,
the survey will remain open to gather a larger number of responses and participants
for more significant results and generalizations.

\begin{table}[ht]
\centering
\begin{adjustbox}{width=0.5\textwidth}
\small
\noindent\begin{tabular}{p{0.17\textwidth}*{20}{|p{0.01\textwidth}}|}
\hline
\multicolumn{0}{|c|}{Work plan}
    & \multicolumn{4}{c|}{January}
    & \multicolumn{4}{c|}{February}
    & \multicolumn{4}{c|}{March}
    & \multicolumn{4}{c|}{April}
    & \multicolumn{4}{c|}{May}
    \\
\hline
\hline
\multicolumn{0}{|l|}{Week}
    & 1 & 2 & 3 & 4 & 1 & 2 & 3 & 4& 1 & 2 & 3 & 4& 1 & 2 & 3 & 4& 1 & 2 & 3 & 4 \\
\hline
% using the on macro to fill in twenty cells as `on'
\multicolumn{0}{|l|}{Discovery and planning}
    & \cellcolor[cmyk]{1,1,0,0}&&&& &&&& &&&& &&&&&&& \\
\hline
\multicolumn{0}{|l|}{Research enquiry}
    & \cellcolor[cmyk]{1,1,0,0} & \cellcolor[cmyk]{1,1,0,0} & \cellcolor[cmyk]{1,1,0,0} & \cellcolor[cmyk]{1,1,0,0} & \cellcolor[cmyk]{1,1,0,0} & \cellcolor[cmyk]{1,1,0,0} & \cellcolor[cmyk]{1,1,0,0} & \cellcolor[cmyk]{1,1,0,0} & \cellcolor[cmyk]{1,1,0,0} & \cellcolor[cmyk]{1,1,0,0} & \cellcolor[cmyk]{1,1,0,0} & \cellcolor[cmyk]{1,1,0,0} &&&& &&&& \\
\hline
\multicolumn{0}{|l|}{State of the art}
    & \cellcolor[cmyk]{1,1,0,0} & \cellcolor[cmyk]{1,1,0,0} & \cellcolor[cmyk]{1,1,0,0} & \cellcolor[cmyk]{1,1,0,0} & \cellcolor[cmyk]{1,1,0,0} & \cellcolor[cmyk]{1,1,0,0} & \cellcolor[cmyk]{1,1,0,0} & \cellcolor[cmyk]{1,1,0,0} &&&& &&&& &&&& \\
\hline
\multicolumn{0}{|l|}{Project requirements}
    & \cellcolor[cmyk]{1,1,0,0} & \cellcolor[cmyk]{1,1,0,0} & \cellcolor[cmyk]{1,1,0,0} & \cellcolor[cmyk]{1,1,0,0} &&&& &&&& &&&& &&&& \\
\hline
\multicolumn{0}{|l|}{Wireframes and user stories}
    &&&& & \cellcolor[cmyk]{1,1,0,0} & \cellcolor[cmyk]{1,1,0,0} & \cellcolor[cmyk]{1,1,0,0} & \cellcolor[cmyk]{1,1,0,0} &&&& &&&& &&&& \\
\hline
\multicolumn{0}{|l|}{Prototyping and refinement}
    &&&&&& && \cellcolor[cmyk]{1,1,0,0} & \cellcolor[cmyk]{1,1,0,0} & \cellcolor[cmyk]{1,1,0,0} & \cellcolor[cmyk]{1,1,0,0} &&&& &&&& &  \\
\hline
\multicolumn{0}{|l|}{Development}
    &&&& &&&& & \cellcolor[cmyk]{1,1,0,0} & \cellcolor[cmyk]{1,1,0,0} & \cellcolor[cmyk]{1,1,0,0} & \cellcolor[cmyk]{1,1,0,0} & \cellcolor[cmyk]{1,1,0,0} & \cellcolor[cmyk]{1,1,0,0} & \cellcolor[cmyk]{1,1,0,0} & \cellcolor[cmyk]{1,1,0,0} & \cellcolor[cmyk]{1,1,0,0} & \cellcolor[cmyk]{1,1,0,0} & \cellcolor[cmyk]{1,1,0,0} & \cellcolor[cmyk]{1,1,0,0} \\
\hline
\multicolumn{0}{|l|}{Tests and iterations}
    &&&& &&&& &&& & \cellcolor[cmyk]{1,1,0,0} & \cellcolor[cmyk]{1,1,0,0} & \cellcolor[cmyk]{1,1,0,0} & \cellcolor[cmyk]{1,1,0,0} & \cellcolor[cmyk]{1,1,0,0} &&&&  \\
\hline
\multicolumn{0}{|l|}{Release and documentation}
    &&&& &&&& &&&& &&&& & \cellcolor[cmyk]{1,1,0,0} & \cellcolor[cmyk]{1,1,0,0} & \cellcolor[cmyk]{1,1,0,0} & \cellcolor[cmyk]{1,1,0,0} \\
\hline
\end{tabular}
\end{adjustbox}
\vspace{1em}
\caption{Work plan timeline}
\label{workchart}
\end{table}

As can be seen in Table \ref{workchart}, the first months will involve the
design of the application and the enquiry of the study followed by the development
of the application and the synthesis of the study, and finally testing and
refinement of the application. Because of the exploratory nature of this
work the application might suffer alterations to the design, specially in
the testing stage, and also depending on the results of the study.

\section{Challenges}

Um dos pontos mais difíceis de realizar nesta tese foi o questionário,
não o facto de construir o questionário mas sim de angariar participantes.
Para além de ser difícil por si só conseguir ter um número de participantes
relativamente alto (algumas centenas, 500 a 1000) para conseguirmos tirar
conclusões com algum grau de confiança elevado, foi difícil de conseguir
com que os possíveis participantes se interessassem no tópico em questão,
pois apesar de parecer que muitas pessoas valorizem muito a sua privacidade
e achem que devem a proteger na prática não se interessam muito. Isto até
pode ser porque muitas pessoas não tenham muito conhecimento a nível da
Internet of Things, e assim acharem que não conseguem responder ao questionário
por ser fora do seu campo de conhecimento, outra razão pode o questionário
parecer um pouco longo, pois demora em média 15 a 20 minutos para responder,
e apesar de até ser um tópico de interesse o investimento de tempo no
questionário pode ser considerado muito elevado. Outro ponto a ter em consideração
em relação ao baixo número de participantes é a forma como o questionário
está escrito e como este foi divulgado, isto é, pode ter sido usado
uma linguagem muito formal ou técnica tanto na construção do mesmo como
também na sua divulgação e juntado ao facto de este ser um tópico muito niche
pode ter "assustando" possíveis participantes. Contudo é de notar que
também na literatura que tem vindo a ser realizada não há um grande foco
na realização de questionários e os que têm sido realizados não têm se
focado somente na Internet of Things e têm também algum incentivo monetário
para os participantes.
