% SPDX-License-Identifier: CC-BY-4.0
%
% Copyright (c) 2023 Nelson Vieira
%
% @author Nelson Vieira <2080511@student.uma.pt>
% @license CC-BY-4.0 <https://creativecommons.org/licenses/by/4.0/legalcode.txt>
\section{Conclusion}\label{section:conclusion}

This work aimed to do an exploratory analysis of privacy in IoT systems.
It proposed a survey to better understand user's knowledge on this subject
and an application that aimed to create more user awareness and better inform
them about their environment, as well as the IoT devices that inhabit it and
how they can respond accordingly.

This work contributed to the overall body of research by compiling and reviewing
other works with the perspective of privacy as a distinct subject matter rather than
an extension of security, as many publications imply. The survey conducted
on the perception of individuals on privacy in IoT systems portrays the majority
viewpoint of portuguese people, since 60\% of participants were portuguese. Additionally,
a mobile application was developed and tested revealing that it performs as it was
initially designed and envisioned since it reaches its purpose on its own without having
to rely on additional platforms.

It is recommended that more attention be placed on user-centric approaches that
examine privacy literacy in IoT systems. According to the literature review and
user testing, there is a significant knowledge gap regarding IoT privacy, particularly
for individuals who have a general lack of technological literacy.

Hopefully, the work conducted on this dissertation will be useful in
supporting researchers going forward and the application developed will be able to
provide greater visibility, thus allowing users to acquire knowledge about
the data being collected and how they can adjust their behaviour or respond
more effectively to protect their privacy rights.
