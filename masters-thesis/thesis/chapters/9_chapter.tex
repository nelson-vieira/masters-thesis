% SPDX-License-Identifier: CC-BY-4.0
%
% Copyright (c) 2023 Nelson Vieira
%
% @author Nelson Vieira <nelson0.vieira@gmail.com>
% @license CC-BY-4.0 <https://creativecommons.org/licenses/by/4.0/legalcode.txt>
\section{Conclusion}\label{section:conclusion}

This work aimed to do an exploratory analysis of privacy in IoT systems.
It proposed a survey to better understand user's knowledge on this subject
and an application that aimed to create more user awareness and better inform
them about their environment, as well as the IoT devices that inhabit it and
how they can respond accordingly.

This work contributes to the overall body of research by compiling and reviewing
other works with the perspective of privacy as a distinct subject matter rather than
an extension of security, as many publications imply. By conducting a survey
on the perception of individuals on privacy in IoT systems from the majority
viewpoint of portuguese people, as 60\% of participants were portuguese. Additionally,
a mobile application was developed that is capable of doing its purpose
on its own without having to rely on additional platforms.

Hopefully, the work conducted on this dissertation will be useful in
supporting researchers going forward and the application developed will be able to
provide greater visibility, thus allowing users to acquire knowledge about
the data being collected and how they can adjust their behaviour or respond
more effectively to protect their privacy rights.
