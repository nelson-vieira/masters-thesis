% SPDX-License-Identifier: CC-BY-4.0
%
% Copyright (c) 2023 Nelson Vieira
%
% @author Nelson Vieira <nelson0.vieira@gmail.com>
% @license CC-BY-4.0 <https://creativecommons.org/licenses/by/4.0/legalcode.txt>

\section{Proposed solutions}

\par This section list seven solutions that emerged from the structured
literature review to improve the gap between privacy and security concepts
among systems and users.

\subsection{Creating new ways for user awareness}

There has been some work done to determine the users awareness of their
actions online regarding their privacy. Skirpan et al. \cite{SkirpanPrivacy}
built an interactive theatre experience, this was created to try to prove
that a simulated experience with a credible privacy problem may encourage
people to take action before actually encountering a catastrophe. The plot
of the play consist in a fledgling tech company that unveiled its revolutionary
AI technology while dealing with a company whistleblower and an untimely
zero-day hack on their system. The public is able to interact with the actors
and influence how the story plays out. Audiences and actors were given the
chance to try on roles, behaviors, and opinions that they would not normally
have access to in ordinary life. The authors had interviews and surveys
done after the plays with audience members however they only did interviews
halfway through production and only a small fraction of the audience actually
participated in this data collection, they also noted that after contacting
people months after the interviews that they did not really changed their
behaviour regarding their privacy rights.

% Skirpan et al. \cite{SkirpanPrivacy} built
% an interactive theatre experience, this was proposed in order to expose privacy
% violations in companies, particularly in the capitalistic world where profit
% is prioritized above all else. In this experiment, the public is able to interact
% with the actors and influence the story of the play; there are various endings
% depending on the public's choices throughout the play; in some endings, the
% company would bury the corruption that was going on; in another ending, a
% team of hackers is abducted. Following the performance, the experiment's crew
% would speak with public members about their experiences and explain
% what it was all about, as well as the members' everyday practices with their data.

% An interactive theatre experience \cite{SkirpanPrivacy}
% was proposed in order to expose privacy malpractices in companies,
% specially in the capitalistic world where profit is prioritize
% above all else, in this experiment the public is able to interact with
% the actors and influence the story of the play, there are various endings depending
% on the public's choices throughout the play, some endings the company would
% bury the corruption that was going on, in another ending a team of hackers
% is able to expose the company's practices to the world. After the play
% the team responsible for the experiment would talk with public members about
% what they experienced and discussed what was it about and the members
% practices with their data on their daily life. After some months
% the team would talk again with the members that were present in the play and
% talk about any changes they have done in the meantime, most said they did
% not change their behavior, one member said that it took more care of The
% information that made available online because it had a bad experience before
% where some private data was exposed that should have not been exposed. All in
% all the experiment did not prove to be a success in changing people's behavior.

\subsection{Legislation}

Some papers seek to improve legislation \cite{WEBER2015618, FabianoInternet}
because otherwise, in their view, privacy rights won't be respected if they
are not enforceable legally, they defend that without the express agreement
of the individual concerned, private information obtained by IoT devices
must not be retained or processed in any form, and necessary procedures
must be taken to guarantee that the data collected is not that of an unrelated
individual. But better protection laws for the user would also create opposition
from most companies that want to extract as much private data from their
users without (m)any restrictions in order to increase their profit margins.

\subsection{Privacy through security}

Sun et al. \cite{SunSecure} design a lightweight communication strategy
for a remote-control system, employing two types of Virtual-Spaces to achieve
the aim of identity announcement and data exchange. They constructed a prototype
system of the scheme and tested it on the Freenet, demonstrating that the
method can effectively resist the influence of flow analysis on communication
anonymity while preserving communication data security.

\subsection{Architecture / Framework Proposals}

Antunes et al. \cite{AntunesFederated} do a SLR on federated learning in
the area of healthcare and make an architecture proposal. The technique
known as federated learning allows for the distributed training of machine
learning models using remotely hosted datasets without the requirement for
data amplification. The fundamental goal of the proposed architecture is
to allow healthcare institutions that have access to sensitive medical information
to use it in distributed data analysis and machine learning research while
ensuring patient confidentiality. Because information transmitted among
institutions need confidentiality guarantees for learning model parameters
and analysis results, the architecture can adopt a number of ways based on
a zero-trust security paradigm \cite{ChenSecurity}. Furthermore, the institutions
develop a learning algorithm verification system that can store and disseminate
manifestos, as well as engage in distributed analytic procedures that need
unanimous agreement from all participants. This study also demonstrates
that previous literature implies that homomorphic encryption and differential
privacy are effective approaches for preventing data breaches without incurring
prohibitively high computing costs.

% The proposed architecture's main goal is to enable healthcare institutions with access
% to private medical datasets to employ them in distributed data analysis and machine
% learning studies without compromising patient confidentiality, as the information that
% is shared among institutions requires confidentiality guarantees for learning model
% parameters and analyses results, the architecture can implement different methodologies
% based on a zero-trust security model [67]. The institutions also implement a learning
% algorithm verification, which can store and provide manifestos and participate in
% distributed analysis mechanisms requiring consensus among all participants. This study
% also finds that Current literature shows that homomorphic encryption and differential
% privacy are techniques with substantial results to avoid data leaks without prohibitive
% computational costs [98].

Opara et al. \cite{opara2022framework} present a system for spotting possible
problems with privacy or security regulations in the early stages of development,
this approach is intended at developers. The paper proposes a domain-specific
ontology for modeling IoT security and privacy policies, a notation for
representing and validating IoT security and privacy policies, a set of
guidelines and rules for detecting IoT policy errors, and a tool for visually
modeling and capturing IoT security policies and discovering policy problems.
Although the framework that is presented is theoretically promising it has
not been tested in a real environment so the effectiveness can't yet be
measured. The authors also do not compare their proposal with others already
available.

% Opara et al. \cite{opara2022framework} propose a framework for detecting potential
% problems with privacy or security policies from the early pre-production stage,
% this framework is aimed at developers. This paper proposes a domain-specific
% ontology for modeling IoT security and privacy policies, a notation for
% representing IoT security and privacy policies and validating the policies
% for inconsistencies, conflicts, ambiguities, and incompleteness, a set of
% guidelines and rules for detecting IoT policy errors, and a tool for
% visually modeling and capturing IoT security policies and discovering
% policy problems.

% This paper's main contributions are: a domain-specific ontology for modeling
% IoT security and privacy policies, a notation for representing IoT security
% and privacy policies and validating the policies for inconsistencies, conflicts,
% ambiguities, and incompleteness, a set of guidelines and rules for
% detecting IoT policy errors, and a tool for visually modeling and capturing
% IoT security policies and discovering problems with the policies.

% The key contributions of this paper include (1) A domain-specific ontology for modeling IoT security and privacy policies, (2) a notation for representing IoT security and privacy policies and validating the policies for inconsistencies, conflicts, ambiguities, and incompleteness, (3) a set of guidelines and rules for detecting IoT policy errors, and 4) a tool for visually modeling and capturing IoT security policies and discovering problems with the policies.

\subsection{Blockchain}

Blockchain is an option to guarantee privacy in IoT because of zero-knowledge
proofs, ring signatures and mixing \cite{PrivacyblockchainWikipedia}.

A zero-knowledge proof is a cryptographic technique that enables one party
(the prover) to demonstrate to another (the verifier) that a certain claim
is true without disclosing any information other than the validity of that
claim. Completeness, soundness, and zero-knowledge are the three requirements
that must be satisfied by a zero-knowledge proof method. Completeness means
that the verifier must be able to confirm that the prover is stating the truth
if the information supplied by the prover is true. Soundness indicates that
the verifier must be given the opportunity to contradict the prover's claims
of speaking the truth if the information provided by the prover is untrue.
Zero-knowledge refers to the need that the method only reveal to the verifier
whether the prover is speaking the truth or not.

Ring signatures create a single, recognizable signature that is used to sign
a transaction by combining a number of partial digital signatures from diverse
users. This group, known as the ring, can be chosen at random from the outputs
that other users have made to the blockchain. A ring signature has the security
property that it should be computationally expensive to determine which
of the group's members' keys was used to produce the signature, this is because
it obfuscates the input side of a transaction. A user's anonymity cannot
be taken away from their signature, and any group of users can act as a signing
group automatically.

Mixing is the process of blending possibly traceable digital assets
with others to obscure the original assets' sources. This is frequently done
by pooling source assets from different inputs for a long period and at random
intervals, then spitting them back out to destination addresses. Since they
are all packed together and then delivered at random intervals, it is very
difficult to pinpoint particular assets. Due to the fact that cryptocurrencies
provide a public record of every transaction, mixers have been developed
to improve cryptocurrency privacy. Because of their emphasis on secrecy,
mixers have been used to launder money using cryptocurrency.

Yu et al. \cite{yu2018blockchain} shows various implementations of blockchain
that provide privacy through security, based on different categories like
data integrity, data sharing and authentication and access control. The
authors use privacy as a proxy for security, they also do not discuss the
weak and strong points of each implementation or make any comparison, they
also do not provide further research questions.

% Ali et al. \cite{AliIoT} propose a software stack of blockchain smart contracts
% and peer-to-peer file storage, to give IoT users authority over
% their data in order to eliminate the need for centralized IoT data
% management. In the proposed `modular consortium' architecture, blockchain
% smart contracts perform access control, while providing
% accountability for both the data owners and the third parties
% whom the users allow access to.

Ali et al. \cite{AliIoT} suggest a software stack that combines peer-to-peer
file sharing with blockchain smart contracts to offer IoT users control
over their data and do away with the necessity for centralized IoT data
management. Blockchain smart contracts are used in the proposed `modular
consortium' architecture to regulate access while establishing responsibility
for both data owners and other parties that users grant access to.

\subsection{Other proposals}

% Zhu et al. \cite{ZhuIntegrating} explores an integrated paradigm called ``hybrid sensing''
% where users interact with a crowdsensing server via a privacy-preserving protocol
% to preserve their anonymity.

Zhu et al. \cite{ZhuIntegrating} present a hybrid sensor system that safeguards
privacy while also monitoring parking availability. The authors merged IoT
sensing with crowdsensing and enhanced it with privacy-preserving methods.
The authors employed physical hazy filters to mask IoT sensors in IoT sensing,
and a cryptographic technique based on cryptographic commitments, zero-knowledge
proofs, and anonymous credentials in crowdsensing. In addition, they used
crowdsourcing to create a machine learning model for parking recognition
in the presence of foggy filters. Their paper included proof-of-concept
prototypes such as a Raspberry Pi system and a mobile app, as well as an
evaluation study of the machine learning model and the effects of crowdsourcing.

% Zhu et al. describe a privacy-preserving hybrid sensor system with a smart parking availability monitoring application. We combined IoT sensing with crowdsensing and augmented it using privacy-protecting approaches. In particular, in IoT sensing, we concealed IoT sensors using physical hazy filters, and in crowdsensing, we used a cryptographic approach based on cryptographic commitments, zero-knowledge proofs, and anonymous credentials. We also used crowdsourcing to build a machine learning model for parking detection under hazy filters. The paper showed proof-of-concept prototypes, including a Raspberry Pi system and a mobile app, as well as an assessment study of the machine learning model and the impacts of crowdsourcing.

% This study offers a hybrid sensor system that protects privacy while also monitoring parking availability.
% We merged IoT sensing with crowdsensing and enhanced it using privacy-preserving techniques.
% We employed physical hazy filters to disguise IoT sensors in IoT sensing, and a cryptographic technique based on cryptographic commitments, zero-knowledge proofs, and anonymous credentials in crowdsensing.
% We also used crowdsourcing to create a machine learning model for parking recognition in the presence of hazy filters.
% The report included proof-of-concept prototypes such as a Raspberry Pi system and a mobile app, as well as an evaluation study of the machine learning model and the effects of crowdsourcing.

% Zhu et al. propose a hybrid sensor system that both protects privacy and monitors parking availability. We combined IoT sensing with crowdsensing and improved it with privacy-preserving approaches. In IoT sensing, we used physical hazy filters to hide IoT sensors, and in crowdsensing, we used a cryptographic approach based on cryptographic commitments, zero-knowledge proofs, and anonymous credentials. Crowdsourcing was also utilized to develop a machine learning model for parking detection in the presence of foggy filters. Proof-of-concept prototypes such as a Raspberry Pi system and a mobile app were included in the paper, as well as an assessment study of the machine learning model and the impacts of crowdsourcing.

% Zhu et al. offer a hybrid sensor system that safeguards privacy while also monitoring parking availability. The researchers merged IoT sensing with crowdsensing and enhanced it with privacy-preserving methods. They employed physical hazy filters to mask IoT sensors in IoT sensing, and a cryptographic technique based on cryptographic commitments, zero-knowledge proofs, and anonymous credentials in crowdsensing. In addition, crowdsourcing was used to create a machine learning model for parking recognition in the presence of foggy filters. The report included proof-of-concept prototypes such as a Raspberry Pi system and a mobile app, as well as an evaluation study of the machine learning model and the effects of crowdsourcing.

\subsection{Privacy Assistants}

There exists a number of privacy assistants in the market. Privacy assistants
have the objective of giving the user flexibility in choosing the preferred
privacy options in available applications, most are used in smartphones,
very few are made for devices in the IoT.

The Carnegie Mellon University CyLab, which is the university's security
and privacy research institute, started developing in 2019 an IoT Infrastructure
that intended to be free of privacy leaks and software covered by their
Secure and Private IoT Initiative 2019, this project would fall under
their main research theme of Trust. In this project they started the design
of a Personalized Privacy Assistant (PPA) \cite{ColnagoInforming}, this
would involve the use of semi-structured interviews with 17 participants
to examine user perceptions of three hypothetical PPA implementations,
each of which is potentially more autonomous, while outlining the advantages
and disadvantages of each implementation. The interviews were divided into
three sections: exploratory, anchoring and the PPA; While the exploratory
phase's purpose was to learn about participants' attitudes and understanding
of IoT, the anchoring phase aimed to normalize participants' basic understanding
of how IoT functions. In order to get people to think about potential privacy
concerns towards the end of the anchoring section, the authors asked participants
about their opinions on data privacy. In the PPA section, it was proposed
the idea of a PPA for IoT as a potential future project. The authors clarified
that the PPA could distinguish between active data requests such as a gadget
asking biometric information from the user's health tracker and passive
data collection such as a smart device with a microphone that could record
people's utterances while they were nearby. The Notification, Recommendation,
and Auto implementations of an IoT PPA were the three that the authors and
attendees discussed. Notification PPAs can determine which adjacent devices
are requesting data and alert users to those devices' presence and requests
so that users can approve or reject each request. Building on notification
PPAs, recommendation PPAs offer consumers advice on how to share their data
based on their preferences. The user's data sharing decisions would be made
by auto PPAs. This would lessen the cognitive load on consumers but also
take away their ability to influence the process. They found that the participants'
attitudes regarding the various implementations were generally favorable,
although they also voiced worries, which varied depending on the degree
of automation. Given the divergent motivations of participants some desired
increased control, while others wished to avoid being overtaken by notifications
and the lack of agreement regarding the optimal PPA implementation.

After the design phase, the institute implemented a privacy assistant (PA) \cite{FengDesign},
the authors called it IoT Assistant. Because the predominant approach of
"notice and choice" for data privacy protection, they decided the PA would
also fall into this approach, but because many systems implement notice
as a form of consent, without sometimes offering choices to the end user,
they also wanted this work to provide a conceptual framework that views
user-centered privacy choice as well as a taxonomy for practitioners to
use when designing meaningful privacy choices for their systems. The authors
define meaningful privacy choices as ``the capabilities provided by digital
systems for users to control different data practices over their personal
data'', They extend the notion of privacy choices with five facets: effectiveness
(the opportunity to establish privacy preferences that precisely and completely
match the data collection and use methods that a user is okay with), efficiency
(the capacity to specify these options with the least amount of effort and
time), user awareness (where significant privacy options should be prominently
and clearly communicated to users), comprehensiveness (users should understand
their options, how they affect the gathering and potential use of their
data, as well as what conclusions might be drawn from this data and the
potential repercussions of these conclusions) and neutrality (meaningful
privacy decisions should not be subject to manipulation or bias). The IoT
Assistant offers four privacy settings, giving end users a variety of alternatives
to better suit their varied privacy preferences and as a result, privacy
options are more effective in the IoT environment. The IoT Assistant acts
as a centralized privacy choice platform by implementing various privacy
options, allowing consumers to more effectively govern their data privacy
in IoT. The three IoT system discovery modes that the IoT Assistant supports
are QR codes, push notifications, and location-based map interfaces. These
discovery tools are probably going to make users more aware of the installed
IoT devices and the privacy options they have. Additionally, the united
viewpoint of the integrated notification and option in the the IoT Assistant
gives succinct yet thorough information regarding IoT data practices to
help users better understand the implications of their privacy choices.
Additionally, the authors work to implement the integrated notice and option
in the IoT Assistant without bias or framing, attempting to offer consumers
a neutral space to execute their privacy choices. Although the authors view
the IoT Assistant as a significant step towards ``meaningful privacy options''
in IoT, this assistant still has many problems, such as the fact that it
is still in its early stages of development and that there hasn't been much
growth given that it was created in 2020 and we are in 2023. Maybe the main
reason this application was not able to be developed further is that the
application itself serves to show the user the data that is already in the
IoT infrastructure that was created before, and as such it is not capable
of identifying new IoT devices without the end users themselves create on
the infrastructure's main webpage \cite{DasPersonalized} a new entry for
the device in question that the user wants to interact with. Another reason
that cripples this application as well as others that seek to provide better
privacy in IoT systems is that many systems do not offer any type of privacy
choices to the end user or to other users that are not the intended end
users but the devices are still collecting data about.

The IoT infrastructure that was developed \cite{DasPersonalized} is built
on an open, distributed design that allows for the deployment and management
of IoT resources to be carried out by any number of actors. Part of this
infrastructure is the Internet of Things Resource Registry, it is a web
platform that enables resource owners to declare not only the place where
a resource is deployed but also data practices like the reason(s) for a
particular data collecting process, the level of detail in the data being
gathered, retention, the recipients of the data, and more. Additionally,
it discloses any user-configurable privacy settings that might be connected
to a particular resource.

\subsection{Sniffers}

IoT sniffers are usually used to detect problems in the networks, they rarely
are used to provide privacy for the users.

The LTEye project \cite{KumarLTE} is an open platform that provides granular
temporal and spatial analytics on the performance of LTE radios without access
to private user data or provider assistance. Despite the presence of multipath,
LTEye uses a revolutionary extension of synthetic aperture radar to communication
signals in order to precisely pinpoint mobile users.

% \subsection{User Awareness}\label{AA}
% There has been some work done to determine the users awareness of
% their actions online in regard to privacy. An interactive theatre experience \cite{ColnagoInforming}
% was proposed in order to expose privacy malpractices in companies,
% specially in the capitalistic world where profit is prioritize
% above all else, in this experiment the public is able to interact with
% the actors and influence the story of the play, there are various endings depending
% on the public's choices throughout the play, some endings the company would
% bury the corruption that was going on, in another ending a team of hackers
% is able to expose the company's practices to the world. After the play
% the team responsible for the experiment would talk with public members about
% what they experienced and discussed what was it about and the members
% practices with their data on their daily life. After some months
% the team would talk again with the members that were present in the play and
% talk about any changes they have done in the meantime, most said they did
% not change their behavior, one member said that it took more care of The
% information that made available online because it had a bad experience before
% where some private data was exposed that should have not been exposed. All in
% all the experiment did not prove to be a success in changing people's behavior.

\subsection{Main Takeaways}

There are two main ways to provide privacy in IoT systems, through security
or using privacy notices, other ways like through legislation or with the
creation/usage of a framework that provides privacy fall into these two
categories. Most of the literature assumes that security and privacy are
synonyms, for example \cite{opara2022framework, FabianoInternet, SunSecure},
and so most of the proposed solutions fall under privacy through security.
The proposed solutions that use privacy notices, like \cite{FengDesign},
are implemented in a way that use other devices like smartphones that provide
the notices themselves, it is hard to provide privacy notices on the IoT
devices themselves because many of these devices do not have a screen or
the screen is too small to provide the necessary information to the user.
Because there are still no standards for implementing privacy notices, and
best practices are scattered throughout the literature, they are mostly
implemented haphazardly, little guidance is given to designers and developers
on how to make a privacy notice design that is sufficient and acceptable
for their particular system and its features. Designers may be unaware of
the numerous possibilities for creating acceptable privacy notifications
and, as a result, do not systematically explore them.

Aleisa and Renaud \cite{aleisa2016privacy} also identify security and privacy
awareness as potential solutions to privacy issues in IoT, but also identify
data minimization, hitchhiking and introspection. Data minimization entails
limiting the collecting of personal information to what is absolutely central
and retaining the data just for as long as is required to satisfy the goal
of the technology's services \cite{ojDirective281}. Hitchhiking \cite{tang2006putting}
is a method of protecting the privacy of users who divulge their location,
applications regard locations as the object of their attention. The fidelity
tradeoff is removed as it is not important to know who is in a certain
location. The introspection \cite{kang2015protection} method examines VM
actions to adequately safeguard users' private information. Every VM's
CPU status, memory contents, network information provided by the hypervisor,
and any malicious software that may be present on the VM are all collected
and analyzed. The privacy of consumers is jeopardized if an IoT device
loses integrity due to a hostile assault.

\subsection{Título subsecção}

\subsubsection{Título subsubsecção}

