% SPDX-License-Identifier: CC-BY-4.0
%
% Copyright (c) 2023 Nelson Vieira
%
% @author Nelson Vieira <nelson0.vieira@gmail.com>
% @license CC-BY-4.0 <https://creativecommons.org/licenses/by/4.0/legalcode.txt>
\section{Challenges}

One of the most difficult tasks to accomplish in this dissertation was the
questionnaire, not the construction of the questionnaire itself but
obtaining participants. Besides the difficulty of obtaining a relatively
large number of participants to draw conclusions with any high degree of
confidence, it was difficult to get potential participants interested in the topic
at hand, because although it appears that many people value their privacy very
highly and believe they should protect it, they do not appear to be very
interested in participating in a survey about it.
This may be because many individuals do not have a high degree of knowledge
about the Internet of Things, and thus feel that they cannot answer the
questionnaire as it is out of their area of expertise. The fact that
it takes on average 15 to 20 minutes to complete the questionnaire suggests
that this may be another contributing factor, despite being an interesting
topic for some.

Another point to take into consideration regarding the low
number of participants is the way the questionnaire is written and how
it was advertised, i.e., a very formal or technical language may have
been used both in the creation of the questionnaire and in its
dissemination, and the fact that this is a very niche topic may have
made possible participants apprehensive. However, it should also be mentioned
that there is not a lot of emphasis on conducting questionnaires in the
literature that has been done, and the ones that have been done have not
just focused on the Internet of Things but also include some form of financial
incentive for the participants. There was no financial incentive for
participation in the questionnaire or the usability test.

\subsection{Lessons Learned}

Tackling privacy issues in IoT systems more complex than initially though.
The Internet of Things have inherent aspects that make it different from
other fields, like the web or mobile, which makes it particularly hard
to apply already proven methods. Take for instance the prevalence of
privacy notices in websites, it is not possible to simply convert this
method and apply it IoT devices as these have different physical shapes and sizes,
different components, different screens and some devices do not even have screens
in which to show these notices, one possible solution in these cases is to
put these notices in other places external to the device but this is still not
a great.

The compilation of works in Section \ref{section:state_of_the_art} helps
to put into perspective the different approaches that have been developed,
open issues, challenges and possible future research paths. It is illuminating
how the literature perceives privacy, with most using it as a synonym of security.
Few works specifically tackle only privacy in Internet of Things, even fewer
are proposing solutions without involving security.

This works' proposal has been created in response to the gap that exists
in privacy literacy solutions that are aimed at the end user.
