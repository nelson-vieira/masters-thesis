% SPDX-License-Identifier: CC-BY-4.0
%
% Copyright (c) 2023 Nelson Vieira
%
% @author Nelson Vieira <nelson0.vieira@gmail.com>
% @license CC-BY-4.0 <https://creativecommons.org/licenses/by/4.0/legalcode.txt>
\section{Privacy Challenges}

IoT is a composed of a complex web of architectures, applications and technologies.
In terms of architectures, it can be decomposed in three layers: the perception
layer, the network layer and the application layer.

The perception layer, also known as the sensor layer, interacts with physical
objects and components via smart devices (RFID, sensors, actuators, and
so on). Its key objectives are to connect objects to the IoT network and
to monitor, collect, and analyze status information about these things using
deployed smart devices. This layer can often be unreliable, for instance
with autonomous vehicles where they find it hard to read road signs or to
predict if certain objects are inanimate or not, but this unreliability
also brings privacy even though some of the data might be unusable. Noise
can also be added in this layer to provide extra privacy.

In the network layer there are many competing networks like ZigBee, Z-Wave,
Bluetooth Low Energy, LoRa, Wi-fi, etc., this layer is fragmented specially
in regards to wireless networks and that makes it very difficult to create
an IoT architecture that can use various networks and have the various
devices communicate with each other, even though interoperability is seen
as a very important factor in IoT. Some of these networks are open standard
protocols while others are proprietary and use different protocols of communication,
use different frequencies, different ranges and different data rates. When
creating an IoT architecture the designers often think of how to solve
specific problems and use what is best for the current needs, and the way
that IoT is fragmented doesn't help in providing progress.

The application layer receives data from the network layer and uses it to
execute essential services or operations. This layer, for example, can provide
the storage service to backup incoming data into a database or the analysis
service to analyze received data in order to predict the future state of
physical devices. This layer encompasses a wide range of applications, each
with its own set of requirements. A few examples are smart grids, smart
transportation, and smart cities.

% In the network layer there are many competing networks like ZigBee, Z-wave
% Bluetooth Low Energy, LoRa, Wi-fi, etc., this layer is fragmented specially
% in regards to wireless networks, and that makes it very difficult to create
% an IoT architecture that can use various networks and have the various
% devices communicate with each other, even though interoperability is seen
% as a very important factor in IoT. Some of these networks are free to use
% while others are proprietary and use different protocols of communication,
% use different frequencies, different ranges and different data rates. When
% creating an IoT architecture the designers often think of how to solve
% specific problems and use what is best for the current needs, and the way
% that IoT is fragmented doesn't help in providing progress.
% In the network layer there are many competing networks like ZigBee, Z-wave
% Bluetooth Low Energy, LoRa, Wi-fi, etc., this layer is fragmented specially
% in regards to wireless networks, and that makes it very difficult to create
% an IoT architecture that can use various networks and have the various
% devices communicate with each other, even though interoperability is seen
% as a very important factor in IoT. Some of these networks are free to use
% while others are proprietary and use different protocols of communication,
% use different frequencies, different ranges and different data rates. When
% creating an IoT architecture the designers often think of how to solve
% specific problems and use what is best for the current needs, and the way
% that IoT is fragmented doesn't help in providing progress.

According to Qu et al. \cite{Qu2018Privacy}, several significant barriers
remain, including the lack of a theoretical foundation, the trade-off optimization
between privacy and data value, and system isomerism over-complexity. Because
there are no mathematical foundations for IoT structure design, IoT system
designs are planned and executed using empirical approaches, which have
limitations in IoT development. Scientific theory and quantitative analysis
must enable trade-off optimization, yet, there are multiple parties with
diverse characteristics and requirements, making this optimization highly
challenging. A plethora of standards and protocols add to the unneeded complexity
of system isomerism. Ensuring effective IoT applications while wasting as
little resources as feasible implies less resources available for privacy
protection, however, lightweight privacy protection cannot fulfill all of
the criteria, and attackers can exploit structural information to launch
several concurrent attacks.

% According to Qu et al. \cite{Qu2018Privacy}, several substantial barriers
% remain, including the lack of a theoretical framework, the trade-off optimization
% between privacy and data utility, and system isomerism over-complexity.
% There are no mathematical foundations for IoT structure design, IoT system
% designs are planned and performed using empirical ways, the shortcomings of
% empirical methods limit IoT progress. To begin with, improving IoT performance
% entirely on human experience is tough. Second, developing privacy protection
% systems is difficult without theoretical guidance. Third, opponents may
% utilize this function to increase the success rate of their attacks. Trade-off
% optimization must be supported by scientific theory and quantitative analysis.
% Qu et al. highlight several key challenges that still need to be overcome
% namely the lack of a theoretical foundation, the trade-off optimization between
% privacy and data utility and system isomerism over-complexity. There are no
% mathematical foundations for IoT structure design, IoT system architectures
% are designed and implemented using empirical methods, the disadvantages of
% empirical methods limit the development of IoT. First, it is not easy to
% optimize IoT performance simply based on human experience. Second, it is
% difficult to implement privacy protection mechanisms without theoretical
% guidance. Third, adversaries can utilize this fact and increase the success
% rates of attacks. Trade-off optimization has to be built on scientific theory
% and quantitative analysis. However, there are multiple parties with dynamic
% characteristics and diversified requirements, which greatly complicates this
% optimization. Moreover, a lack of theoretical foundation causes non-uniform
% quantitative measurements and thereby introduces uncertainty into trade-off
% optimization. A massive number of standards and protocols facilitate the over-complication
% of the isomerism of systems. An over-complicated isomerism causes inconveniences
% to communication and system integration. Ensuring effective IoT applications
% without wasting too many resources leaves fewer resources for privacy protection.
% Nevertheless, lightweight privacy protection cannot satisfy all the requirements.
% In addition, adversaries can utilize the structural information to launch various
% and continuous attacks.
