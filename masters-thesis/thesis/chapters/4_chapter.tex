% SPDX-License-Identifier: CC-BY-4.0
%
% Copyright (c) 2023 Nelson Vieira
%
% @author Nelson Vieira <nelson0.vieira@gmail.com>
% @license CC-BY-4.0 <https://creativecommons.org/licenses/by/4.0/legalcode.txt>
\section{Challenges}

One of the most difficult points to accomplish in this thesis was the
questionnaire, not the construction of the questionnaire itself but
getting participants. Besides being difficult in itself to get a relatively
high number of participants (a few hundred at least)
to be able to draw conclusions with any high degree of confidence, it was
difficult to get the potential participants interested in the topic
at hand, because although it seems that many people value their privacy very
highly and think they should protect it, in practice it seems they
are not very interested. This may even be because many people do not have much knowledge
about the Internet of Things, and thus feel that they cannot answer the
questionnaire because it is out of their field of knowledge, another
reason may be that the questionnaire seems a little long, because it
takes on average 15 to 20 minutes to answer, and despite being a topic
of interest the time investment in the questionnaire may be considered
too high. Another point to take into consideration regarding the low
number of participants is the way the questionnaire is written and how
it was advertised, i.e., a very formal or technical language may have
been used both in the construction of the questionnaire and in its
dissemination, and the fact that this is a very niche topic may have
"scared" possible participants. However, it should be noted that also
in the literature that has been carried out there is not a great focus
on conducting questionnaires and the ones that have been conducted have
not only focused on the Internet of Things and also have some monetary
incentive for the participants.
