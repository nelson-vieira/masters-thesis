% SPDX-License-Identifier: CC-BY-4.0
%
% Copyright (c) 2023 Nelson Vieira
%
% @author Nelson Vieira <nelson0.vieira@gmail.com>
% @license CC-BY-4.0 <https://creativecommons.org/licenses/by/4.0/legalcode.txt>

%%% CHAPTER 1/Capitulo 1
%
\section{Introduction}

Privacy as we know it is a somewhat recent concept \cite{vincent2016privacy, moore2017privacy},
before the digital age there was barely any notion of privacy for most
people. For many centuries most people used to reside in small communities
where they were continuously involved in one another's lives. Even more
recent is the idea that privacy is a crucial component of personal security,
in contrast to the undeniable necessity of public security, including the
requirement for guarded walls and closed doors. Long seen as a luxury, privacy
is still usually regarded as a good to have rather than an essential
requirement, even though it is acknowledged as a human right, as present
in article 12 of the Universal Declaration of Human Rights \cite{RooseveltUniversal}:
``No one shall be subjected to arbitrary interference with his privacy,
family, home or correspondence, nor to attacks upon his honour and reputation.
Everyone has the right to the protection of the law against such interference
or attacks''. Privacy can be defined \cite{InternationalWhat, SpiekermannEngineering}
as the right to govern how personal information and data is collected, stored,
and used, it frequently involves handling sensitive information with care,
and as such, organizations must be open and honest about the kind of data
they plan to gather, why they need it, and where and with whom they plan
to share it. Users should have the right to control their shared information.

This definition can cause some confusion with the idea of security \cite{HIVDifference}
and although privacy and security are interconnected, security involves
measures taken to safeguard data from risk, threat or danger, it frequently
alludes to safety. It is the practice of keeping users' personal information
and data safe and preventing unauthorized access to it. The primary contrast
between privacy and security is that the former deals with personal information
to individuals and how they want their data used and maintained, whilst
the latter deals with its protection from possible threats. Security can
exist without privacy, but the opposite is not true. For managing sensitive
and personal data, privacy and computer security are equally crucial. Users
should be aware of the internal procedures regarding the collection, processing,
retention, and sharing of personal information.

Concerns about digital privacy have been growing \cite{emami2019exploring, park2022personal, zhang2022peer}
in the last few years, especially after the Anonymous decentralized hacker
group cyber attacks, WikiLeaks and Snowden's leaked top secret documents
from United State's National Security Agency. These concerns can be noted
with the increase of written literature on the subject, when searching for
terms like ``privacy'', ``online privacy'', ``digital privacy'' in Google
Scholar, ACM Digital Library or Science Direct it can be seen that, in the
last 5 years, it returns about 5000000, 650000 and 80000 documents respectively,
including articles, books, conference papers etc.

Most research has focused on the web, while privacy in IoT systems has not
been explored as much. Because IoT devices are becoming more prevalent,
new methods of communicating, gathering, and analyzing data emerge.
Because there is already a substantial quantity of research focusing on
web privacy rather than IoT privacy, it is a lot more fertile ground to
explore the issue of privacy in the context of the IoT.

\textit{Internet of Things} is a term that first appeared in the 1990s,
and it may be linked to Mark Weiser's paper on ubiquitous computing \cite{weiser1991computer}
and the growth of devices of all sizes that communicate with one another
to do various tasks, making Weiser's dream a reality. The first use of the
term \textit{Internet of Things} was in 1999 by British technology pioneer
Kevin Ashton \cite{KevinThat}, executive director of the Auto-ID Center
at Massachusetts Institute of Technology (MIT), to describe a system in
which items may be connected to the internet by sensors. He came up with
the phrase while giving a presentation for Procter \& Gamble to highlight
the value of linking Radio-Frequency Identification (RFID) tags used in
corporate supply chains to the internet in order to count and track goods
without the need for human assistance. These devices are used in various
applications, starting at home \cite{marikyan2019systematic} with thermostats,
fridges, microwaves, etc, moving on to smart cars \cite{arena2020overview},
the educational system \cite{al2020survey}, our clothes and our watches \cite{niknejad2020comprehensive}
and even into outer space \cite{AkyildizInternet}. IoT resources may include
IoT equipment (like smart home assistants and autonomous vehicles), IoT
services (like video analytics services linked to smart cameras and indoor
position tracking systems), or IoT apps (like smart TV remote apps) that
track and use information about us. Internet of Things is now widely used
to describe situations in which a range of objects, gadgets, sensors, and
ordinary items are connected to the internet and have computational capabilities.

The idea of using computers and networks in order to monitor and manage
devices is nothing new, despite the term \textit{Internet of Things} being
relatively recent.
% By the late 1970s, technologies for remotely monitoring electricity
% grid meters through telephone lines were already in use in the corporate
% sector \cite{}.
Wireless technology improvements in the 1990s permitted the widespread
adoption of corporate and industrial machine-to-machine (M2M) solutions
for equipment monitoring and operation. Many early M2M solutions, on the
other hand, relied on proprietary purpose-built networks or industry-specific
standards rather than internet standards. To connect devices other than
computers to the internet is not a new concept. A Coke machine at Carnegie
Mellon University's Computer Science Department \cite{EverhartInteresting}
was the first ubiquitous device to be linked to the internet. The system,
which was created in 1982, remotely observed the out of stock lights on
the pressing buttons of the vending machine and broadcast the state of each
row of the vending machine on the network so that it could be accessed using
the Name/Finger protocol through a terminal. In 1990, a toaster that could
be turned on and off over the internet that was created by John Romkey \cite{RomkeyToast},
was demonstrated at the Interop Internet Networking show.

% \cite{rose2015internet}
% The term "Internet of Things" (IoT) was first used in 1999 by British technology
% pioneer Kevin Ashton to describe a system in which objects in the physical world
% could be connected to the Internet by sensors.12 Ashton coined the term to
% illustrate the power of connecting Radio-Frequency Identification (RFID) tags13
% used in corporate supply chains to the Internet in order to count and track goods
% without the need for human intervention. Today, the Internet of Things has become
% a popular term for describing scenarios in which Internet connectivity and computing
% capability extend to a variety of objects, devices, sensors, and everyday items.

% While the term "Internet of Things" is relatively new, the concept of combining
% computers and networks to monitor and control devices has been around for decades.
% By the late 1970s, for example, systems for remotely monitoring meters on the
% electrical grid via telephone lines were already in commercial use. In the
% 1990s, advances in wireless technology allowed "machine-to-machine" (M2M)
% enterprise and industrial solutions for equipment monitoring and operation to
% become widespread. Many of these early M2M solutions, however, were based on closed
% purpose-built networks and proprietary or industry-specific standards, rather than
% on Internet Protocol (IP)-based networks and Internet standards. Using IP to connect
% devices other than computers to the Internet is not a new idea. The first Internet
% "device"—an IP-enabled toaster that could be turned on and off over the Internet—was
% featured at an Internet conference in 1990. Over the next several years, other
% "things" were IP-enabled, including a soda machine at Carnegie Mellon University in
% the US and a coffee pot in the Trojan Room at the University of Cambridge in the UK
% (which remained Internet-connected until 2001). From these whimsical beginnings, a
% robust field of research and development into "smart object networking" helped
% create the foundation for today's Internet of Things.

The Internet of Things can be defined as: ``An open and comprehensive network
of intelligent objects that have the capacity to auto-organize, share information,
data and resources, reacting and acting in face of situations and changes
in the environment'' \cite{madakam2015internet}.

IoT is one of the fastest growing technologies \cite{MohammadState}, it
is predicted that it will grow into the trillions of devices by 2030 \cite{SarawiInternet},
and with this expansion new security vulnerabilities and data gathering
dangers appear, the lack of security in these devices makes them ideal targets
for privacy violations and inadequate customer disclosure of device capabilities
and data practices aggravates privacy and security issues.

Privacy in IoT systems in not seen as a crucial factor in development \cite{alhirabi2021security}.
Specific standards for privacy options have been imposed by data privacy
regulations including the General Data Protection Regulation (GDPR) and
California Consumer Privacy Act (CCPA), but even these regulations have
been criticized \cite{peloquin2020disruptive, gladis2022weaponizing, gentile2022deficient, green2022flaws, byun2019privacy}.

\subsection{Título subsecção}

ABC

\subsection{Título subsecção}

ABC

\subsection{Estrutura do Documento}
