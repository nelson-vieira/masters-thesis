% SPDX-License-Identifier: CC-BY-4.0
%
% Copyright (c) 2023 Nelson Vieira
%
% @author Nelson Vieira <2080511@student.uma.pt>
% @contributor Mary Barreto <mary.barreto@staff.uma.pt>
% @license CC-BY-4.0 <https://creativecommons.org/licenses/by/4.0/legalcode.txt>
\section{Introduction} \label{introduction}

Privacy as we know it is a somewhat recent concept \cite{vincent2016privacy, moore2017privacy},
before the digital age there was barely any notion of privacy for most
people. For many centuries most people used to reside in small communities
where they were continuously involved in one another's lives. Even more
recent is the concept that privacy is a crucial component of personal security,
in contrast to the necessity of public security. Privacy has traditionally been
considered a luxury and is still frequently recognized as nicety as opposed to an
essential need, even though it is acknowledged as a human right, as present
in article 12 of the Universal Declaration of Human Rights \cite{RooseveltUniversal}:
``No one shall be subjected to arbitrary interference with his privacy,
family, home or correspondence, nor to attacks upon his honour and reputation.
Everyone has the right to the protection of the law against such interference
or attacks''. Additionally, the right to privacy is recognized in more than 120 national
constitutions \cite{constitute2023constitutions}. Privacy can be defined \cite{InternationalWhat, SpiekermannEngineering}
as the right to govern how personal information and data is collected, stored,
and used, it frequently involves handling sensitive information with care,
and as such, organizations must be open and honest about the kind of data
they plan to gather, why they need it, and where and with whom they plan
to share it. Users should have the right to control their shared information.

This definition can cause some confusion with the idea of security \cite{HIVDifference}
and although privacy and security are interconnected, security involves
measures taken to safeguard data from risk, threat or danger, it frequently
alludes to safety. It is the practice of keeping users' personal information
and data safe and preventing unauthorized access to it. The primary contrast
between privacy and security is that the former deals with personal information
to individuals and how they want their data used and maintained, whilst
the latter deals with its protection from possible threats. Security can
exist without privacy, but the opposite is not true. For managing sensitive
and personal data, privacy and computer security are equally crucial. Users
should be aware of the internal procedures regarding the collection, processing,
retention, and sharing of personal information.

Concerns about digital privacy have been growing \cite{emami2019exploring, park2022personal, zhang2022peer}
in the last few years, especially after the Anonymous decentralized hacker
group cyber attacks, WikiLeaks and Snowden's leaked top secret documents
from United State's National Security Agency. These concerns can be noted
with the increase of written literature on the subject, when searching for
terms like ``privacy'', ``online privacy'', ``digital privacy'' in Google
Scholar, ACM Digital Library or Science Direct it can be seen that, in the
last 5 years, it returns about 5 000 000, 650 000 and 80 000 documents respectively,
including articles, books, conference papers etc.

Privacy has become such an important concern in the digital age in which
we live because most technology organizations heavily rely on
the advertising sector \cite{duan2022pricing}, which in turn depends on customer
data. Following the
2000s, many organizations offered their services to the general public without
charging the user, however, because these organizations depend on revenue to
stay afloat, the user is ultimately treated as a commodity for their data
because they are not required to pay to access the service.

\textit{Internet of Things} \DTLassign{acronyms}{14}{\acronym=Acronym}(\hyperlink{\acronym}{\acronym}) is a term that first appeared in the 1990s,
and it may be linked to Mark Weiser's paper on ubiquitous computing \cite{weiser1991computer}
and the growth of devices of all sizes that communicate with one another
to do various tasks, making Weiser's dream a reality. The first use of the
term \textit{Internet of Things} was in 1999 by British technology pioneer
Kevin Ashton \cite{KevinThat}, executive director of the Auto-ID Center
at Massachusetts Institute of Technology
\DTLassign{acronyms}{17}{\acronym=Acronym}(\hyperlink{\acronym}{\acronym}),
to describe a system in
which items may be connected to the internet by sensors. He came up with
the phrase while giving a presentation for Procter \& Gamble to highlight
the value of linking Radio-Frequency Identification \DTLassign{acronyms}{23}{\acronym=Acronym}(\hyperlink{\acronym}{\acronym}) tags used in
corporate supply chains to the internet in order to count and track goods
without the need for human assistance. These devices are used in various
applications, starting at home \cite{marikyan2019systematic} with thermostats,
fridges, microwaves, etc, moving on to smart cars \cite{arena2020overview},
the educational system \cite{al2020survey}, our clothes and our watches \cite{niknejad2020comprehensive}
and even into outer space \cite{AkyildizInternet}. \DTLassign{acronyms}{14}{\acronym=Acronym}\hyperlink{\acronym}{\acronym} resources may include
\DTLassign{acronyms}{14}{\acronym=Acronym}\hyperlink{\acronym}{\acronym} equipment (like smart home assistants and autonomous vehicles), \DTLassign{acronyms}{14}{\acronym=Acronym}\hyperlink{\acronym}{\acronym}
services (like video analytics services linked to smart cameras and indoor
position tracking systems), or \DTLassign{acronyms}{14}{\acronym=Acronym}\hyperlink{\acronym}{\acronym} apps (like smart TV remote apps) that
track and use information about us. Internet of Things is now widely used
to describe situations in which a range of objects, gadgets, sensors, and
ordinary items are connected to the internet and have computational capabilities.

The Internet of Things can be defined as: ``An open and comprehensive network
of intelligent objects that have the capacity to auto-organize, share information,
data and resources, reacting and acting in face of situations and changes
in the environment'' \cite{madakam2015internet}.

The idea of using computers and networks in order to monitor and manage
devices is nothing new, despite the term \textit{Internet of Things} being
relatively recent.
% By the late 1970s, technologies for remotely monitoring electricity
% grid meters through telephone lines were already in use in the corporate
% sector \cite{}.
Wireless technology improvements in the 1990s permitted the widespread
adoption of corporate and industrial machine-to-machine \DTLassign{acronyms}{16}{\acronym=Acronym}(\hyperlink{\acronym}{\acronym}) solutions
for equipment monitoring and operation. Many early \DTLassign{acronyms}{16}{\acronym=Acronym}\hyperlink{\acronym}{\acronym} solutions, on the
other hand, relied on proprietary purpose-built networks or industry-specific
standards rather than internet standards. To connect devices other than
computers to the internet is not a new concept. A Coke machine at Carnegie
Mellon University's Computer Science Department \cite{EverhartInteresting}
was the first ubiquitous device to be linked to the internet. The system,
which was created in 1982, remotely observed the out-of-stock lights on
the pressing buttons of the vending machine and broadcast the state of each
row of the vending machine on the network so that it could be accessed using
the Name/Finger protocol through a terminal. In 1990, a toaster that could
be turned on and off over the internet that was created by John Romkey \cite{RomkeyToast},
was demonstrated at the Interop Internet Networking show.

\DTLassign{acronyms}{14}{\acronym=Acronym}\hyperlink{\acronym}{\acronym} is one of the fastest growing technologies \cite{MohammadState}, it
is predicted that it will grow into the trillions of devices by 2030 \cite{SarawiInternet},
and with this expansion new security vulnerabilities and data gathering
threats appear, making these devices an ideal target for
privacy violations and inadequate customer disclosure of device capabilities
and data practices aggravates privacy and security issues.

When the earliest computers were created, privacy was not even considered
a concern because they were utilized for basic calculations, it was only in
the decades that followed \cite{hoffman1969computers}, as computers became connected to one another that privacy gradually came to
the forefront. In 1973, the US Department of Health, Education, and Welfare
published \textit{Records, Computers and the Rights of Citizens, Report of the
Secretary's Advisory Committee on Automated Personal Data Systems} \cite{hew1973records},
one of the first documents on digital privacy and an important first step that would
form the basis for modern privacy legislation. In 1977 a revision of privacy
policies would be published by the Privacy Protection Study Commission \cite{united1977personal}.
Meanwhile, in the 1980s, computers were becoming more ubiquitous in workplaces
and increasing popularity in people's homes, sparking debate regarding digital
privacy. With the introduction of the World Wide Web at the start of the following
decade, privacy concerns began to increase.

Privacy in \DTLassign{acronyms}{14}{\acronym=Acronym}\hyperlink{\acronym}{\acronym} systems in not seen as a crucial factor in the design and
development stages \cite{alhirabi2021security}, instead emphasis is placed
on enhancing system quality, providing management controls, and maximising productivity.
Specific standards for privacy options have been imposed by data privacy
regulations including the General Data Protection Regulation \DTLassign{acronyms}{9}{\acronym=Acronym}(\hyperlink{\acronym}{\acronym}) and
California Consumer Privacy Act \DTLassign{acronyms}{3}{\acronym=Acronym}(\hyperlink{\acronym}{\acronym}), but even these regulations have
been criticized \cite{peloquin2020disruptive, gladis2022weaponizing, gentile2022deficient, green2022flaws, byun2019privacy}.

Privacy is considered to be one of the most important concerns in \DTLassign{acronyms}{14}{\acronym=Acronym}\hyperlink{\acronym}{\acronym} by
individuals, according to a study conducted by Mozilla \cite{Jen2017ten}
with nearly 190,000 participants, some papers \cite{khan2021issues, MOHAMMADZADEH2018124}
also consider privacy a primary issue alongside security while others also highlight
energy efficiency \cite{sisinni2018industrial}.

The primary motivation for writing this dissertation was to investigate
the dissonance between individuals high regard for their privacy
and their behaviours and to determine whether this happens because individuals
are lacking the knowledge needed to make better and informed decisions.
The focus on \DTLassign{acronyms}{14}{\acronym=Acronym}\hyperlink{\acronym}{\acronym} was chosen since there is not as much research being done to
explore privacy issues as there is for the web. As \DTLassign{acronyms}{14}{\acronym=Acronym}\hyperlink{\acronym}{\acronym} devices are becoming
more prevalent, new methods of communicating,
gathering, and analysing data emerge.
It is much more fruitful to investigate the issue of privacy in the setting
of the \DTLassign{acronyms}{14}{\acronym=Acronym}\hyperlink{\acronym}{\acronym} because there is already a sizeable body of research focusing on
web or mobile privacy as opposed to \DTLassign{acronyms}{14}{\acronym=Acronym}\hyperlink{\acronym}{\acronym} privacy. This dissertation will mainly
focus on privacy concerns in \DTLassign{acronyms}{14}{\acronym=Acronym}\hyperlink{\acronym}{\acronym} systems in a holistic manner in order to present
fresh perspectives and serve as a repository for the collected understanding of
these issues, challenges and addressed gaps.

By reviewing previous papers from the perspective of privacy as a subject matter,
rather than as a synonym for security, conducting a survey on individuals' privacy literacy,
and developing a mobile application as a tool to empower individuals' privacy decisions,
this study contributes to the corpus of research as a whole.

The approach for this dissertation is formulated in two phases, the first of
which is a systematic literature review \DTLassign{acronyms}{26}{\acronym=Acronym}(\hyperlink{\acronym}{\acronym}), and the second is a questionnaire, and
a mobile application. These stages are discussed
in the several chapters that make up the dissertation's structure: Chapter \ref{section:state_of_the_art}
is composed of a systematic literature review that focuses on gathering the state
of the art in terms of the most relevant approaches to privacy issues in \DTLassign{acronyms}{14}{\acronym=Acronym}\hyperlink{\acronym}{\acronym}, as
well challenges inherit in \DTLassign{acronyms}{14}{\acronym=Acronym}\hyperlink{\acronym}{\acronym} and other relevant topics that help give an
overview of \DTLassign{acronyms}{14}{\acronym=Acronym}\hyperlink{\acronym}{\acronym} privacy. Finally, following a review of each work, the main lessons
learned are outlined, the most significant gaps in the literature are highlighted,
and suggestions for future research that could be undertaken to tackle \DTLassign{acronyms}{14}{\acronym=Acronym}\hyperlink{\acronym}{\acronym} privacy
are provided. This chapter aims to address these two research questions \DTLassign{acronyms}{25}{\acronym=Acronym}(\hyperlink{\acronym}{\acronym}s):\\

% \vspace{5mm}
% \textbf{Phase 1 (Literature review):} \\

\textbf{\DTLassign{acronyms}{25}{\acronym=Acronym}\hyperlink{\acronym}{\acronym}1:} What approaches are currently being considered to address privacy
issues in \DTLassign{acronyms}{14}{\acronym=Acronym}\hyperlink{\acronym}{\acronym}?

\textbf{\DTLassign{acronyms}{25}{\acronym=Acronym}\hyperlink{\acronym}{\acronym}2:} What issues are prevalent in \DTLassign{acronyms}{14}{\acronym=Acronym}\hyperlink{\acronym}{\acronym} that make it challenging to
protect individuals' privacy? \\

Chapter \ref{section:methodology} describes the methodology of this work, it is composed of a
questionnaire and a mobile application. The questionnaire aimed to gather
the general privacy concerns of individuals, their online habits, their
understanding of privacy concepts and their relation with \DTLassign{acronyms}{14}{\acronym=Acronym}\hyperlink{\acronym}{\acronym} devices.
The mobile application was designed and developed based on the findings
of the \DTLassign{acronyms}{26}{\acronym=Acronym}\hyperlink{\acronym}{\acronym} and the questionnaire, it seeks to help users broaden their
knowledge of \DTLassign{acronyms}{14}{\acronym=Acronym}\hyperlink{\acronym}{\acronym} and privacy with the goal to enable them to make more
informed decisions. Two research questions will be considered in this
chapter:\\

% \vspace{5mm}
% \textbf{Phase 2 (Survey and application):} \\

\textbf{\DTLassign{acronyms}{25}{\acronym=Acronym}\hyperlink{\acronym}{\acronym}3:} What are the perceptions of individuals on online privacy?

\textbf{\DTLassign{acronyms}{25}{\acronym=Acronym}\hyperlink{\acronym}{\acronym}4:} How can users be empowered to protect their privacy in \DTLassign{acronyms}{14}{\acronym=Acronym}\hyperlink{\acronym}{\acronym} systems?\\
% \vspace{5mm}

Chapter \ref{section:application} delves into the comprehensive creation process
of the mobile application from the requirements gathering to prototypes, development
and finally the usability tests.
The results of the questionnaire and the usability tests of the application are
covered in Chapter \ref{section:results}, while the discussion of these results is reserved for
Chapter \ref{section:discussion}, which contains the main findings for both stages, \DTLassign{acronyms}{26}{\acronym=Acronym}\hyperlink{\acronym}{\acronym} and the
application evaluation, and also addresses the research questions posed for phase 2 of
this work. Subsequently, Chapter \ref{section:challenges} describes challenges and lessons
learned while conducting the overall work, from the \DTLassign{acronyms}{26}{\acronym=Acronym}\hyperlink{\acronym}{\acronym} to the application
development. The following Chapter \ref{section:future_work} lists recommendations and future approaches and areas of work.
Finally, Chapter \ref{section:conclusion} presents the conclusion of the work.
