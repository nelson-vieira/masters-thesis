% SPDX-License-Identifier: CC-BY-4.0
%
% Copyright (c) 2023 Nelson Vieira
%
% @author Nelson Vieira <2080511@student.uma.pt>
% @contributor Mary Barreto <mary.barreto@staff.uma.pt>
% @license CC-BY-4.0 <https://creativecommons.org/licenses/by/4.0/legalcode.txt>
\section{Future work}\label{section:future_work}

The usability tests were performed with a beta version of the application,
and the results are promising but since the application was not tested in a production environment,
the next step would be to release it on Apple Store and Google Play to
gauge how users would react to it.

Although there are existing hardware solutions that can detect some devices
on particular networks, like ZigBee or Bluetooth LE, namely IoT sniffers
and there exist some georeferencing applications that try to pinpoint certain
IoT devices, there is still a need for some kind of device or framework
that is network agnostic and can detect where the devices are located and
what kind of data the IoT devices that are around it are collecting. This
device should also be capable of informing users about the privacy notices
of the devices and what can the users do to safeguard their personal data.
The IoT sniffers that are available are primarily used in the detection
of problems in the communication of devices in the network or to solve problems
of interoperability between different IoT networks. There are many obstacles
that prevent the creation of such a device, and the fact that nothing similar
to it is currently in existence may be due to either a lack of interest on the
part of users or researchers or the complexity of the task being greater than
the advantages.

\subsection{Recommendations}

Based on current research trends and on the findings of the user tests that were
conducted during the course of this work, there is a big focus on certain
fields, such as security and AI while others are left unexplored.

A research path that is underdeveloped is privacy literacy in IoT systems,
while privacy is explored along with security and other adjacent fields, this aspect of IoT
is still in its infancy, but should be further researched as there is a
clear lack of knowledge by most individuals. This particular situation prevents
them from forming decisions that benefit them in the long term. Organizations
already exploit this fact and bad actors can also take advantage of this, as
has already happened.

Another aspect of IoT that deserves more attention
is the application of privacy in the design and development phases of new
IoT systems. There are many constraints to take into account when creating a
new system like functionality, security, viability of the system, and business
requirements among others, privacy should also be seen as a crucial aspect
of any system.

Interoperability is considered to be very important in the IoT,
but most systems do not naturally support this concept because many devices use particular
communication networks that do not easily communicate with other types of networks,
necessitating the use of convoluted methods to achieve the desired interoperability.

One other aspect to take into account, the user should always be at the center
of any system because systems are created by people to be used by people, this is not
always considered in the myriad of things that comprise the work of developers
and researchers, with worries about time, productivity, and the over
engineering of many aspects of work life. Human-Computer interaction should
be ever present in IoT.

To summarize, these are the characteristics of IoT that deserve more attention:

\begin{itemize}
    \item[$\bullet$]
    Privacy literacy in IoT systems;
    \item[$\bullet$]
    Application of privacy in the design/development of IoT systems;
    \item[$\bullet$]
    Interoperability standards;
    \item[$\bullet$]
    User-centric approaches to IoT privacy.
\end{itemize}
