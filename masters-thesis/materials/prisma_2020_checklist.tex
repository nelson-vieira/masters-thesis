% SPDX-License-Identifier: CC-BY-4.0
%
% Copyright (c) 2023 Nelson Vieira
%
% @author Nelson Vieira <2080511@student.uma.pt>
% @license CC-BY-4.0 <https://creativecommons.org/licenses/by/4.0/legalcode.txt>
\documentclass[landscape]{article}
\usepackage{cite}
\usepackage[T1]{fontenc}
\usepackage[utf8]{inputenc}
\usepackage[table]{xcolor}
\usepackage{lscape}
\usepackage{longtable}
\usepackage{amsmath,amssymb,amsfonts}
\usepackage{algorithmic}
\usepackage{graphicx}
\usepackage[bookmarks=true]{hyperref}
\hypersetup{
    colorlinks=false,
}
\def\UrlBreaks{\do\/\do-}
\def\BibTeX{{\rm B\kern-.05em{\sc i\kern-.025em b}\kern-.08em
    T\kern-.1667em\lower.7ex\hbox{E}\kern-.125emX}}
\usepackage[margin=1cm]{geometry}
\pagenumbering{gobble}
\usepackage{multicol,multirow}
\newenvironment{mytabular}[1][1]{%
    \renewcommand*{\arraystretch}{#1}%
    \tabular%
}{%
    \endtabular
}
\def\myversion{1.0}

\title{PRISMA 2020 Checklist}
\author{Nelson Vieira\\
\textit{Faculty of Exact Sciences and Engineering} \\
\textit{University of Madeira}\\
Funchal, Portugal \\
2080511@student.uma.pt
}

\begin{document}

\maketitle

\begin{longtable}{|p{2.5cm}|p{0.5cm}|p{18cm}|p{3cm}|}
    \hline
    \rowcolor{lightgray!5}
    \textbf{Section and Topic} & \textbf{Item} & \textbf{Checklist Item} & \textbf{Location where item is reported} \\
    \hline
    \multicolumn{4}{l}{TITLE} \\
    \hline
    Title & 1 & Privacy in the Internet of Things: Fostering User Empowerment through Digital Literacy - Systematic Literature Review & \\
    \hline \hline
    \multicolumn{4}{l}{ABSTRACT} \\
    \hline
    Abstract & 2 & \textbf{Background}: The Internet of Things (IoT) has the potential to revolutionize the way we interact with the world around us, but it also raises significant concerns about privacy. \newline\textbf{Objective}: This paper proposes the development of an application to foster privacy literacy among users in regards to the IoT. \newline\textbf{Methods}: The application will provide users with information about the privacy implications of using IoT devices, as well as tools and resources to help them protect their personal information. Results: Through this application, we aim to empower users to take control of their privacy in the IoT era. Conclusions: By fostering privacy literacy among users, we can help mitigate the risks associated with the widespread adoption of IoT technology and ensure that its benefits are realized in a safe and secure manner. & \\
    \hline \hline
    \multicolumn{4}{l}{INTRODUCTION} \\
    \hline
    Rationale & 3 & Describe the rationale for the review in the context of existing knowledge. & \\
    \hline
    Objectives & 4 & Provide an explicit statement of the objective(s) or question(s) the review addresses. & \\
    \hline \hline
    \multicolumn{4}{l}{METHODS} \\
    \hline
    Eligibility Criteria & 5 & Specify the inclusion and exclusion criteria for the review and how studies were grouped for the syntheses & \\
    \hline
    Information sources & 6 & Specify all databases, registers, websites, organisations, reference lists and other sources searched or consulted to identify studies. Specify the date when each source was last searched or consulted. & \\
    \hline
    Search strategy & 7 & Keywords: privacy, regulation, legislation, "internet of things", "privacy paradox", IoT, framework, architecture, "differential privacy", "privacy literacy", "user knowledge", "individual knowledge", "privacy rights", "consumer rights", stakeholder, responsibility, awareness, expectation, consent, "privacy notice", blockchain, holochain, security, challenge, challenges, risks, "privacy assistant", sniffer, healthcare, "smart home", "smart house", "smart city", "smart vehicle", "autonomous vehicle", "machine learning", "deep learning", SLR, "systematic literature review", "mapping literature review", "personalized privacy", network, networks, sustainability, approaches, proposals, edge, fog, cloud, "big data", "artificial intelligence of things", survey, study, opportunity, opportunities, "smart agriculture", "trust management", "access control", "data collection", authorization, "privacy concerns", "privacy issues", "privacy challenges", trust, "privacy management", "federated learning", "smart city"; "Privacy regulation proposals for internet of things". "Measuring individual knowledge regarding privacy in the internet of things" "What is the user knowledge regarding privacy in the internet of things" & \\
    \hline
    Selection process & 8 &  Specify the methods used to decide whether a study met the inclusion criteria of the review, including how many reviewers screened each record and each report retrieved, whether they worked independently, and if applicable, details of automation tools used in the process. & \\
    \hline
    Data collection process & 9 &  Specify the methods used to collect data from reports, including how many reviewers collected data from each report, whether they worked independently, any processes for obtaining or confirming data from study investigators, and if applicable, details of automation tools used in the process. & \\
    \hline
    \multirow{2}{2.5cm}{Data items}& 10a & List and define all outcomes for which data were sought. Specify whether all results that were compatible with each outcome domain in each study were sought (e.g. for all measures, time points, analyses), and if not, the methods used to decide which results to collect.& \\
    \cline{2-4}
    & 10b & List and define all other variables for which data were sought (e.g. participant and intervention characteristics, funding sources). Describe any assumptions made about any missing or unclear information. & \\
    \hline
    Study risk of bias assessment &11  & Specify the methods used to assess risk of bias in the included studies, including details of the tool(s) used, how many reviewers assessed each study and whether they worked independently, and if applicable, details of automation tools used in the process. & \\
    \hline
    Effect measures  &  12 & Specify for each outcome the effect measure(s) (e.g. risk ratio, mean difference) used in the synthesis or presentation of results & \\
    \hline
    \multirow{6}{2.5cm}{Synthesis methods} & 13a & Describe the processes used to decide which studies were eligible for each synthesis (e.g. tabulating the study intervention characteristics and comparing against the planned groups for each synthesis (item \#5)). & \\
    \cline{2-4}
    & 13b & Describe any methods required to prepare the data for presentation or synthesis, such as handling of missing summary statistics, or data conversions. & \\
    \cline{2-4}
    & 13c & Describe any methods used to tabulate or visually display results of individual studies and syntheses. & \\
    \cline{2-4}
    & 13d & Describe any methods used to synthesize results and provide a rationale for the choice(s). If meta-analysis was performed, describe the model(s), method(s) to identify the presence and extent of statistical heterogeneity, and software package(s) used. & \\
    \cline{2-4}
    & 13e & Describe any methods used to explore possible causes of heterogeneity among study results (e.g. subgroup analysis, meta-regression).  & \\
    \cline{2-4}
    & 13f & Describe any sensitivity analyses conducted to assess robustness of the synthesized results.  & \\
    \hline
    Reporting bias assessment & 14 & Describe any methods used to assess risk of bias due to missing results in a synthesis (arising from reporting biases). & \\
    \hline
    Certainty assessment & 15 & Describe any methods used to assess certainty (or confidence) in the body of evidence for an outcome. & \\
    \hline
    \multicolumn{4}{l}{RESULTS} \\
    \hline
     \multirow{2}{2.5cm}{Study selection}& 16a & Describe the results of the search and selection process, from the number of records identified in the search to the number of studies included in the review, ideally using a flow diagram. & \\
     \cline{2-4}
     & 16b & Cite studies that might appear to meet the inclusion criteria, but which were excluded, and explain why they were excluded. & \\
     \hline
     Study Characteristics &  17&  Cite each included study and present its characteristics.& \\
    \hline
     Risk of bias in studies & 18  & Present assessments of risk of bias for each included study. & \\
     \hline
     Results of individual studies & 19 &  For all outcomes, present, for each study: (a) summary statistics for each group (where appropriate) and (b) an effect estimate and its precision (e.g. confidence/credible interval), ideally using structured tables or plots.& \\
     \hline
     \multirow{4}{2.5cm}{Results of syntheses}& 20a & For each synthesis, briefly summarise the characteristics and risk of bias among contributing studies & \\
     \cline{2-4}
    & 20b & Present results of all statistical syntheses conducted. If meta-analysis was done, present for each the summary estimate and its precision (e.g. confidence/credible interval) and measures of statistical heterogeneity. If comparing groups, describe the direction of the effect. & \\
    \cline{2-4}
    & 20c & Present results of all investigations of possible causes of heterogeneity among study results.  & \\
    \cline{2-4}
    & 20d & Present results of all sensitivity analyses conducted to assess the robustness of the synthesized results. & \\
    \hline
     Reporting biases&21 & Present assessments of risk of bias due to missing results (arising from reporting biases) for each synthesis assessed.& \\
     \hline
     Certainty of evidence &22 & Present assessments of certainty (or confidence) in the body of evidence for each outcome assessed. & \\
     \hline \hline
    \multicolumn{4}{l}{DISCUSSION} \\
    \hline
    \multirow{4}{2.5cm}{Discussion}&23a  & Provide a general interpretation of the results in the context of other evidence. & \\
    \cline{2-4}
     & 23b &Discuss any limitations of the evidence included in the review. & \\
     \cline{2-4}
     & 23c &Discuss any limitations of the review processes used. & \\
     \cline{2-4}
     &  23d&Discuss implications of the results for practice, policy, and future research. & \\

    \hline \hline
    \multicolumn{4}{l}{OTHER INFORMATION} \\
    \hline
    \multirow{3}{2.5cm}{Registration and protocol}& 24a  &  Provide registration information for the review, including register name and registration number, or state that the review was not registered.& \\
    \cline{2-4}
     &  24b&Indicate where the review protocol can be accessed, or state that a protocol was not prepared. & \\
     \cline{2-4}
     &  24c& Describe and explain any amendments to information provided at registration or in the protocol.& \\
     \hline
     Support& 25 &Describe sources of financial or non-financial support for the review, and the role of the funders or sponsors in the review. & \\ \hline
     Competing interests &  26& Declare any competing interests of review authors.& \\ \hline
      Availability of data, code and other materials & 27 & Report which of the following are publicly available and where they can be found: template data collection forms; data extracted from included studies; data used for all analyses; analytic code; any other materials used in the review.& \\
    \hline
    \multicolumn{4}{l}{TITLE} \\
    \hline
    Title & 1 & Identify the report as a systematic review. & \\
    \hline \hline
    \multicolumn{4}{l}{BACKGROUND} \\
    \hline
    Objectives & 2 &Provide an explicit statement of the main objective(s) or question(s) the review addresses. & \\
    \hline \hline
    \multicolumn{4}{l}{METHODS} \\
    \hline
    Eligibility criteria &3 &Specify the inclusion and exclusion criteria for the review. & \\
    \hline
    Information sources & 4&Specify the information sources (e.g. databases, registers) used to identify studies and the date when each was last searched. & \\
    \hline
    Risk of bias& 5& Specify the methods used to assess risk of bias in the included studies. & \\
    \hline
    Synthesis of results & 6& Specify the methods used to present and synthesise results.& \\
    \hline \hline
    \multicolumn{4}{l}{RESULTS} \\
    \hline
    Included studies & 7& Give the total number of included studies and participants and summarise relevant characteristics of studies.& \\
    \hline
    Synthesis of results & 8&Present results for main outcomes, preferably indicating the number of included studies and participants for each. If meta-analysis was done, report the summary estimate and confidence/credible interval. If comparing groups, indicate the direction of the effect (i.e. which group is favoured). & \\
    \hline \hline
    \multicolumn{4}{l}{DISCUSSION} \\
    \hline
    Limitations of evidence& 9 &Provide a brief summary of the limitations of the evidence included in the review (e.g. study risk of bias, inconsistency and imprecision). & \\
    \hline
    Interpretation&  10& Provide a general interpretation of the results and important implications.& \\
    \hline \hline
    \multicolumn{4}{l}{OTHER} \\
    \hline
    Funding& 11& Not applicable. & \\
    \hline
    Registration& 12& Provide the register name and registration number.& \\
    \hline
    \caption{PRISMA 2020 Checklist}
    \label{table:prisma_long}
\end{longtable}

% \clearpage

% \begin{table}
%     \caption{PRISMA 2020 Checklist}
%     \label{table:prisma}
%     {\small
%     \begin{mytabular}[1.2]{|p{2.5cm}|p{0.5cm}|p{18cm}|p{3cm}|}
%         \hline
%         \rowcolor{lightgray}
%         Section and Topic & Item  & Checklist Item & Location where item is reported \\
%         \hline
%         \multicolumn{4}{l}{TITLE} \\
%         \hline
%         Title & 1 & Privacy in the Internet of Things: Fostering User Empowerment through Digital Literacy - Systematic Literature Review & \\
%         \hline \hline
%         \multicolumn{4}{l}{ABSTRACT} \\
%         \hline
%         Abstract & 2 & \textbf{Background}: The Internet of Things (IoT) has the potential to revolutionize the way we interact with the world around us, but it also raises significant concerns about privacy. \newline\textbf{Objective}: This paper proposes the development of an application to foster privacy literacy among users in regards to the IoT. \newline\textbf{Methods}: The application will provide users with information about the privacy implications of using IoT devices, as well as tools and resources to help them protect their personal information. Results: Through this application, we aim to empower users to take control of their privacy in the IoT era. Conclusions: By fostering privacy literacy among users, we can help mitigate the risks associated with the widespread adoption of IoT technology and ensure that its benefits are realized in a safe and secure manner. & \\
%         \hline \hline
%         \multicolumn{4}{l}{INTRODUCTION} \\
%         \hline
%         Rationale & 3 & Describe the rationale for the review in the context of existing knowledge. & \\
%         \hline
%         Objectives & 4 & Provide an explicit statement of the objective(s) or question(s) the review addresses. & \\
%         \hline \hline
%         \multicolumn{4}{l}{METHODS} \\
%         \hline
%         Eligibility Criteria & 5 & Specify the inclusion and exclusion criteria for the review and how studies were grouped for the syntheses & \\
%         \hline
%         Information sources & 6 & Specify all databases, registers, websites, organisations, reference lists and other sources searched or consulted to identify studies. Specify the date when each source was last searched or consulted. & \\
%         \hline
%         Search strategy & 7 & Keywords: privacy, regulation, legislation, "internet of things", "privacy paradox", IoT, framework, architecture, "differential privacy", "privacy literacy", "user knowledge", "individual knowledge", "privacy rights", "consumer rights", stakeholder, responsibility, awareness, expectation, consent, "privacy notice", blockchain, holochain, security, challenge, challenges, risks, "privacy assistant", sniffer, healthcare, "smart home", "smart house", "smart city", "smart vehicle", "autonomous vehicle", "machine learning", "deep learning", SLR, "systematic literature review", "mapping literature review", "personalized privacy", network, networks, sustainability, approaches, proposals, edge, fog, cloud, "big data", "artificial intelligence of things", survey, study, opportunity, opportunities, "smart agriculture", "trust management", "access control", "data collection", authorization, "privacy concerns", "privacy issues", "privacy challenges", trust, "privacy management", "federated learning", "smart city"; "Privacy regulation proposals for internet of things". "Measuring individual knowledge regarding privacy in the internet of things" "What is the user knowledge regarding privacy in the internet of things" & \\
%         \hline
%         Selection process & 8 &  Specify the methods used to decide whether a study met the inclusion criteria of the review, including how many reviewers screened each record and each report retrieved, whether they worked independently, and if applicable, details of automation tools used in the process. & \\
%         \hline
%         Data collection process & 9 &  Specify the methods used to collect data from reports, including how many reviewers collected data from each report, whether they worked independently, any processes for obtaining or confirming data from study investigators, and if applicable, details of automation tools used in the process. & \\
%         \hline
%         \multirow{2}{2.5cm}{Data items}& 10a & List and define all outcomes for which data were sought. Specify whether all results that were compatible with each outcome domain in each study were sought (e.g. for all measures, time points, analyses), and if not, the methods used to decide which results to collect.& \\
%         \cline{2-4}
%         & 10b & List and define all other variables for which data were sought (e.g. participant and intervention characteristics, funding sources). Describe any assumptions made about any missing or unclear information. & \\
%         \hline
%         Study risk of bias assessment &11  & Specify the methods used to assess risk of bias in the included studies, including details of the tool(s) used, how many reviewers assessed each study and whether they worked independently, and if applicable, details of automation tools used in the process. & \\
%         \hline
%         Effect measures  &  12 & Specify for each outcome the effect measure(s) (e.g. risk ratio, mean difference) used in the synthesis or presentation of results & \\
%         \hline
%         \multirow{6}{2.5cm}{Synthesis methods} & 13a & Describe the processes used to decide which studies were eligible for each synthesis (e.g. tabulating the study intervention characteristics and comparing against the planned groups for each synthesis (item \#5)). & \\
%         \cline{2-4}
%         & 13b & Describe any methods required to prepare the data for presentation or synthesis, such as handling of missing summary statistics, or data conversions. & \\
%         \cline{2-4}
%         & 13c & Describe any methods used to tabulate or visually display results of individual studies and syntheses. & \\
%         \cline{2-4}
%         & 13d & Describe any methods used to synthesize results and provide a rationale for the choice(s). If meta-analysis was performed, describe the model(s), method(s) to identify the presence and extent of statistical heterogeneity, and software package(s) used. & \\
%         \cline{2-4}
%         & 13e & Describe any methods used to explore possible causes of heterogeneity among study results (e.g. subgroup analysis, meta-regression).  & \\
%         \cline{2-4}
%         & 13f & Describe any sensitivity analyses conducted to assess robustness of the synthesized results.  & \\
%         \hline
%         Reporting bias assessment & 14 & Describe any methods used to assess risk of bias due to missing results in a synthesis (arising from reporting biases). & \\
%         \hline
%         Certainty assessment & 15 & Describe any methods used to assess certainty (or confidence) in the body of evidence for an outcome. & \\
%         \hline \hline
%     \end{mytabular}}
% \end{table}

% \begin{table}
%     \caption{PRISMA 2020 Checklist, Cont'd.}
%     \label{table:prisma2}
%     {\small
%     \begin{mytabular}[1.25]{|p{2.5cm}|p{0.5cm}|p{18cm}|p{3cm}|}

%     \hline
%     \rowcolor{lightgray}
%     Section and Topic & Item  & Checklist Item & Location where item is reported \\
%     \hline
%     \multicolumn{4}{l}{RESULTS} \\
%     \hline
%      \multirow{2}{2.5cm}{Study selection}& 16a & Describe the results of the search and selection process, from the number of records identified in the search to the number of studies included in the review, ideally using a flow diagram. & \\
%      \cline{2-4}
%      & 16b & Cite studies that might appear to meet the inclusion criteria, but which were excluded, and explain why they were excluded. & \\
%      \hline
%      Study Characteristics &  17&  Cite each included study and present its characteristics.& \\
%     \hline
%      Risk of bias in studies & 18  & Present assessments of risk of bias for each included study. & \\
%      \hline
%      Results of individual studies & 19 &  For all outcomes, present, for each study: (a) summary statistics for each group (where appropriate) and (b) an effect estimate and its precision (e.g. confidence/credible interval), ideally using structured tables or plots.& \\
%      \hline
%      \multirow{4}{2.5cm}{Results of syntheses}& 20a & For each synthesis, briefly summarise the characteristics and risk of bias among contributing studies & \\
%      \cline{2-4}
%     & 20b & Present results of all statistical syntheses conducted. If meta-analysis was done, present for each the summary estimate and its precision (e.g. confidence/credible interval) and measures of statistical heterogeneity. If comparing groups, describe the direction of the effect. & \\
%     \cline{2-4}
%     & 20c & Present results of all investigations of possible causes of heterogeneity among study results.  & \\
%     \cline{2-4}
%     & 20d & Present results of all sensitivity analyses conducted to assess the robustness of the synthesized results. & \\
%     \hline
%      Reporting biases&21 & Present assessments of risk of bias due to missing results (arising from reporting biases) for each synthesis assessed.& \\
%      \hline
%      Certainty of evidence &22 & Present assessments of certainty (or confidence) in the body of evidence for each outcome assessed. & \\
%      \hline \hline
%     \multicolumn{4}{l}{DISCUSSION} \\
%     \hline
%     \multirow{4}{2.5cm}{Discussion}&23a  & Provide a general interpretation of the results in the context of other evidence. & \\
%     \cline{2-4}
%      & 23b &Discuss any limitations of the evidence included in the review. & \\
%      \cline{2-4}
%      & 23c &Discuss any limitations of the review processes used. & \\
%      \cline{2-4}
%      &  23d&Discuss implications of the results for practice, policy, and future research. & \\

%     \hline \hline
%     \multicolumn{4}{l}{OTHER INFORMATION} \\
%     \hline
%     \multirow{3}{2.5cm}{Registration and protocol}& 24a  &  Provide registration information for the review, including register name and registration number, or state that the review was not registered.& \\
%     \cline{2-4}
%      &  24b&Indicate where the review protocol can be accessed, or state that a protocol was not prepared. & \\
%      \cline{2-4}
%      &  24c& Describe and explain any amendments to information provided at registration or in the protocol.& \\
%      \hline
%      Support& 25 &Describe sources of financial or non-financial support for the review, and the role of the funders or sponsors in the review. & \\ \hline
%      Competing interests &  26& Declare any competing interests of review authors.& \\ \hline
%       Availability of data, code and other materials & 27 & Report which of the following are publicly available and where they can be found: template data collection forms; data extracted from included studies; data used for all analyses; analytic code; any other materials used in the review.& \\
     
%     \hline \hline
%     \end{mytabular}}
% \end{table}

% \begin{table}
%     \caption{PRISMA 2020 Abstract Checklist}
%     \label{table:prismaabstract}

%     \begin{mytabular}[1.25]{|p{3cm}|p{0.6cm}|p{18cm}|p{3cm}|}

%     \hline
%     \rowcolor{lightgray}
%     Section and Topic & Item  & Checklist Item & Reported (Yes/No) \\
%     \hline
%     \multicolumn{4}{l}{TITLE} \\
%     \hline
%     Title & 1 & Identify the report as a systematic review. & \\
%     \hline \hline
%     \multicolumn{4}{l}{BACKGROUND} \\
%     \hline
%     Objectives & 2 &Provide an explicit statement of the main objective(s) or question(s) the review addresses. & \\
%     \hline \hline
%     \multicolumn{4}{l}{METHODS} \\
%     \hline
%     Eligibility criteria &3 &Specify the inclusion and exclusion criteria for the review. & \\
%     \hline
%     Information sources & 4&Specify the information sources (e.g. databases, registers) used to identify studies and the date when each was last searched. & \\
%     \hline
%     Risk of bias& 5& Specify the methods used to assess risk of bias in the included studies. & \\
%     \hline
%     Synthesis of results & 6& Specify the methods used to present and synthesise results.& \\
%     \hline \hline
%     \multicolumn{4}{l}{RESULTS} \\
%     \hline
%     Included studies & 7& Give the total number of included studies and participants and summarise relevant characteristics of studies.& \\
%     \hline
%     Synthesis of results & 8&Present results for main outcomes, preferably indicating the number of included studies and participants for each. If meta-analysis was done, report the summary estimate and confidence/credible interval. If comparing groups, indicate the direction of the effect (i.e. which group is favoured). & \\
%     \hline \hline
%     \multicolumn{4}{l}{DISCUSSION} \\
%     \hline
%     Limitations of evidence& 9 &Provide a brief summary of the limitations of the evidence included in the review (e.g. study risk of bias, inconsistency and imprecision). & \\
%     \hline
%     Interpretation&  10& Provide a general interpretation of the results and important implications.& \\
%     \hline \hline
%     \multicolumn{4}{l}{OTHER} \\
%     \hline
%     Funding& 11& Not applicable. & \\
%     \hline
%     Registration& 12& Provide the register name and registration number.& \\
%     \hline \hline

%     \hline \hline
%     \end{mytabular}
% \end{table}

\end{document}
